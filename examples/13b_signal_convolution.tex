% Example 13b: Signal Convolution (Rect * Rect)
% Sliding window convolution of two rectangular pulses.
% Shows f(tau), g(t-tau), and the resulting convolution integral.
% Parameter: \PARAM (time shift t, -1.5 to 3.5)
% Range: -1.5 to 3.5, recommended 60 frames
% Difficulty: advanced
% Features demonstrated: pgfplots, filled regions, multiple synchronized plots
\documentclass[tikz]{standalone}
\usepackage{pgfplots}
\usepackage{xcolor}
\pgfplotsset{compat=1.18}

% USC Brand Colors
\definecolor{garnet}{HTML}{73000A}
\definecolor{rose}{HTML}{CC2E40}
\definecolor{atlantic}{HTML}{466A9F}
\definecolor{congaree}{HTML}{1F414D}
\definecolor{horseshoe}{HTML}{65780B}
\definecolor{grass}{HTML}{CED318}
\definecolor{honeycomb}{HTML}{A49137}
\definecolor{warmgrey}{HTML}{676156}
\definecolor{sandstorm}{HTML}{FFF2E3}

\begin{document}
\begin{tikzpicture}
  \useasboundingbox (-2bp, -2bp) rectangle (258bp, 234bp);
  \pgfmathsetmacro{\tshift}{\PARAM}

  % --- Top plot: f(tau) and g(t - tau) ---
  \begin{axis}[
    name=topplot,
    at={(0,4.5cm)},
    width=10cm, height=4.5cm,
    domain=-2:4,
    samples=200,
    axis lines=middle,
    xlabel={$\tau$},
    ylabel={},
    ymin=-0.3, ymax=1.5,
    xmin=-2, xmax=4,
    grid=major,
    grid style={warmgrey!20},
    title={\small Signals: $f(\tau)$ and $g(t - \tau)$},
    every axis title/.style={at={(0.5,1.1)}, font=\small},
    legend pos=north east,
    legend style={font=\tiny},
    clip=false,
  ]
    % f(tau): rectangular pulse from 0 to 1, height 1
    \addplot[garnet, very thick]
      coordinates {(-2,0) (0,0) (0,1) (1,1) (1,0) (4,0)};
    \addlegendentry{$f(\tau)$}

    % g(t - tau): rectangular pulse, shifted copy from t-1 to t
    % g(x) = rect on [0, 1], so g(t - tau) = 1 when t-1 < tau < t
    \addplot[atlantic, very thick]
      coordinates {
        (-2, 0)
        ({\tshift - 1}, 0)
        ({\tshift - 1}, 1)
        ({\tshift}, 1)
        ({\tshift}, 0)
        (4, 0)
      };
    \addlegendentry{$g(t - \tau)$}

    % Shade overlap region: both are 1 in the overlap, so it's a rectangle
    % f=1 on [0,1], g=1 on [t-1,t]
    % Overlap: [max(0, t-1), min(1, t)]
    \pgfmathsetmacro{\overlapLeft}{max(0, \tshift - 1)}
    \pgfmathsetmacro{\overlapRight}{min(1, \tshift)}
    \pgfmathparse{\overlapLeft < \overlapRight ? 1 : 0}
    \ifnum\pgfmathresult=1
      \fill[honeycomb!50, opacity=0.3]
        (axis cs:\overlapLeft, 0) rectangle (axis cs:\overlapRight, 1);
    \fi
  \end{axis}

  % --- Bottom plot: convolution result (f * g)(t) ---
  \begin{axis}[
    at={(0,0)},
    width=10cm, height=4.5cm,
    domain=-2:4,
    samples=300,
    axis lines=middle,
    xlabel={$t$},
    ylabel={$(f * g)(t)$},
    ymin=-0.3, ymax=1.5,
    xmin=-2, xmax=4,
    grid=major,
    grid style={warmgrey!20},
    title={\small Convolution: $(f * g)(t)$},
    every axis title/.style={at={(0.5,1.1)}, font=\small},
    clip=false,
  ]
    % Analytical convolution of rect[0,1] * rect[0,1]:
    % t in [0, 1]:  t
    % t in [1, 2]:  2 - t
    \addplot[warmgrey!60, very thick, smooth] {
      (x >= 0 && x < 1) * (x) +
      (x >= 1 && x < 2) * (2 - x)
    };

    % Traced portion up to current t
    \addplot[garnet, very thick, smooth, domain=-2:\tshift] {
      (x >= 0 && x < 1) * (x) +
      (x >= 1 && x < 2) * (2 - x)
    };

    % Current point marker
    \pgfmathsetmacro{\convval}{%
      (\tshift >= 0 && \tshift < 1) * (\tshift) +
      (\tshift >= 1 && \tshift < 2) * (2 - \tshift)
    }
    \addplot[only marks, mark=*, mark size=3pt, garnet]
      coordinates {(\tshift, \convval)};

    % Vertical dashed line at current t
    \draw[dashed, garnet!50, thin] (axis cs:\tshift, 0) -- (axis cs:\tshift, \convval);

    \node[anchor=south west, font=\small] at (axis cs:-1.8, 1.2)
      {$t = \pgfmathprintnumber[fixed, precision=2]{\tshift}$};
  \end{axis}
\end{tikzpicture}
\end{document}
