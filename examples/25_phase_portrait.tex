% Example 25: Phase Portrait Animation
% Phase portrait of dx/dt = [[-a, b], [-b, -a]]*x with stability transitions.
% Parameter: \PARAM (real part of eigenvalue a, range -1.0..2.0)
% Range: -1.0 to 2.0, recommended 30 frames
% Difficulty: advanced
% Features demonstrated: phase portrait, ODE, nullclines, eigenvalue inset,
%   trajectory opacity fading, energy-level circles, flow particles

\documentclass[border=8pt,tikz]{standalone}
\usepackage{tikz}
\usepackage{pgfplots}
\usepackage{amsmath}
\usetikzlibrary{arrows.meta}
\pgfplotsset{compat=1.18}

% USC Brand Colors
\definecolor{garnet}{HTML}{73000A}
\definecolor{rose}{HTML}{CC2E40}
\definecolor{atlantic}{HTML}{466A9F}
\definecolor{congaree}{HTML}{1F414D}
\definecolor{horseshoe}{HTML}{65780B}
\definecolor{grass}{HTML}{CED318}
\definecolor{honeycomb}{HTML}{A49137}
\definecolor{warmgrey}{HTML}{676156}
\definecolor{sandstorm}{HTML}{FFF2E3}

\begin{document}
\begin{tikzpicture}[>=Stealth]

\pgfmathsetmacro{\aval}{\PARAM}
\pgfmathsetmacro{\bval}{2.0}

% Axes
\draw[->, thick, black!60] (-4.2, 0) -- (4.2, 0) node[right, font=\small]{$x_1$};
\draw[->, thick, black!60] (0, -4.2) -- (0, 4.2) node[above, font=\small]{$x_2$};

% Grid
\foreach \v in {-4,-3,-2,-1,1,2,3,4} {
    \draw[warmgrey!12, thin] (\v, -4) -- (\v, 4);
    \draw[warmgrey!12, thin] (-4, \v) -- (4, \v);
}

% Energy-level circles (Lyapunov contours V = x1^2 + x2^2 = const)
\foreach \rr in {1.0, 2.0, 3.0} {
    \draw[dashed, warmgrey!20, very thin] (0,0) circle (\rr);
}

% Nullclines
% dx1/dt = 0: -a*x1 + b*x2 = 0  =>  x2 = (a/b)*x1
% dx2/dt = 0: -b*x1 - a*x2 = 0  =>  x2 = -(b/a)*x1 (when a != 0)
\pgfmathsetmacro{\slopeNullVal}{\aval/max(abs(\aval),0.001)*abs(\aval)/\bval}
\pgfmathsetmacro{\absAval}{abs(\aval)}
% Pre-compute all nullcline coordinates
\pgfmathsetmacro{\slopeN}{(\absAval > 0.05) ? \aval/\bval : 0}
\pgfmathsetmacro{\slopeNtwo}{(\absAval > 0.05) ? -\bval/\aval : -100}
\pgfmathsetmacro{\absSlopeN}{abs(\slopeN)}
\pgfmathsetmacro{\nDenom}{max(\absSlopeN+1,1)}
\pgfmathsetmacro{\nXa}{-4/\nDenom}
\pgfmathsetmacro{\nYa}{-4*\slopeN/\nDenom}
\pgfmathsetmacro{\nXb}{4/\nDenom}
\pgfmathsetmacro{\nYb}{4*\slopeN/\nDenom}
\pgfmathsetmacro{\nLabelY}{2.5*\slopeN}
\pgfmathsetmacro{\sNtwoClamp}{min(max(\slopeNtwo, -10), 10)}
\pgfmathsetmacro{\absSNtwo}{abs(\sNtwoClamp)}
\pgfmathsetmacro{\nDenomB}{max(\absSNtwo+1,1)}
\pgfmathsetmacro{\nXc}{-4/\nDenomB}
\pgfmathsetmacro{\nYc}{-4*\sNtwoClamp/\nDenomB}
\pgfmathsetmacro{\nXd}{4/\nDenomB}
\pgfmathsetmacro{\nYd}{4*\sNtwoClamp/\nDenomB}
\pgfmathsetmacro{\nLabelXe}{-2.5/\nDenomB}
\pgfmathsetmacro{\nLabelYe}{-2.5*\sNtwoClamp/\nDenomB}
\pgfmathtruncatemacro{\showNull}{(\absAval > 0.05) ? 1 : 0}
\ifnum\showNull=1
    \draw[rose, thick, dashed] (\nXa, \nYa) -- (\nXb, \nYb);
    \node[font=\tiny, rose, anchor=south west] at (2.5, \nLabelY)
        {$\dot{x}_1\!=\!0$};
    \draw[horseshoe, thick, dashed] (\nXc, \nYc) -- (\nXd, \nYd);
    \node[font=\tiny, horseshoe, anchor=north east]
        at (\nLabelXe, \nLabelYe)
        {$\dot{x}_2\!=\!0$};
\fi

% Vector field
\foreach \xi in {-3.5, -2.5, -1.5, -0.5, 0.5, 1.5, 2.5, 3.5} {
    \foreach \yi in {-3.5, -2.5, -1.5, -0.5, 0.5, 1.5, 2.5, 3.5} {
        \pgfmathsetmacro{\dx}{-\aval*\xi + \bval*\yi}
        \pgfmathsetmacro{\dy}{-\bval*\xi - \aval*\yi}
        \pgfmathsetmacro{\mag}{sqrt(\dx*\dx + \dy*\dy)}
        \pgfmathsetmacro{\scale}{min(0.35, 0.35*\mag/3)}
        \pgfmathparse{\mag > 0.1 ? 1 : 0}
        \ifnum\pgfmathresult=1
            \pgfmathsetmacro{\ndx}{\dx/\mag*\scale}
            \pgfmathsetmacro{\ndy}{\dy/\mag*\scale}
            \draw[->, warmgrey!45, thin] (\xi, \yi) -- ({\xi+\ndx}, {\yi+\ndy});
        \fi
    }
}

% Trajectories with opacity fading
\foreach \thetaStart in {0, 60, 120, 180, 240, 300} {
    \pgfmathsetmacro{\R}{3.0}
    % Draw in segments with varying opacity (clamp radius to avoid overflow)
    \foreach \seg in {0, 1, 2, ..., 9} {
        \pgfmathsetmacro{\tA}{\seg * 0.628}
        \pgfmathsetmacro{\tB}{(\seg + 1) * 0.628}
        \pgfmathsetmacro{\radiusAtB}{exp(-\aval*\tB) * \R}
        \pgfmathparse{\radiusAtB < 50 ? 1 : 0}
        \ifnum\pgfmathresult=1
            \pgfmathsetmacro{\opac}{max(0.9 - \seg*0.08, 0.15)}
            \draw[atlantic, thick, opacity=\opac, samples=20,
                  domain=\tA:\tB, variable=\t]
                plot ({min(max(exp(-\aval*\t) * \R * cos(\bval*\t*180/3.14159 + \thetaStart),-20),20)},
                      {min(max(exp(-\aval*\t) * \R * sin(\bval*\t*180/3.14159 + \thetaStart),-20),20)});
        \fi
    }
    % Initial condition marker
    \filldraw[garnet]
        ({\R * cos(\thetaStart)}, {\R * sin(\thetaStart)}) circle (2.5pt);

    % Flow particle at t=2 (clamp to avoid overflow for unstable spirals)
    \pgfmathsetmacro{\tflow}{2.0}
    \pgfmathsetmacro{\rFlow}{exp(-\aval*\tflow) * \R}
    \pgfmathsetmacro{\fpx}{(\rFlow < 50) ? \rFlow * cos(\bval*\tflow*180/3.14159 + \thetaStart) : 999}
    \pgfmathsetmacro{\fpy}{(\rFlow < 50) ? \rFlow * sin(\bval*\tflow*180/3.14159 + \thetaStart) : 999}
    \pgfmathparse{(abs(\fpx) < 4.0) && (abs(\fpy) < 4.0) ? 1 : 0}
    \ifnum\pgfmathresult=1
        \filldraw[congaree] (\fpx, \fpy) circle (1.5pt);
    \fi
}

% Equilibrium
\filldraw[black] (0, 0) circle (3pt);

% Stability label
\pgfmathparse{\aval > 0.05 ? 1 : (\aval < -0.05 ? -1 : 0)}
\pgfmathtruncatemacro{\stabType}{\pgfmathresult}

\ifnum\stabType=1
    \node[anchor=north east, font=\footnotesize, fill=horseshoe!10, inner sep=3pt,
          draw=horseshoe]
        at (4.0, 4.0) {Stable spiral};
\fi
\ifnum\stabType=-1
    \node[anchor=north east, font=\footnotesize, fill=garnet!10, inner sep=3pt,
          draw=garnet]
        at (4.0, 4.0) {Unstable spiral};
\fi
\ifnum\stabType=0
    \node[anchor=north east, font=\footnotesize, fill=honeycomb!10, inner sep=3pt,
          draw=honeycomb]
        at (4.0, 4.0) {Center};
\fi

% ODE label
\node[anchor=south west, font=\footnotesize, fill=white, inner sep=3pt]
    at (-4.0, -4.0)
    {$\dot{\mathbf{x}} = \begin{bmatrix} -a & b \\ -b & -a \end{bmatrix} \mathbf{x}$,\;
     $a = \pgfmathprintnumber[fixed,precision=2]{\aval}$};

% Eigenvalue s-plane inset
\begin{axis}[
    at={(4.0cm, -3.0cm)}, anchor=south east,
    width=3cm, height=3cm,
    xmin=-2.5, xmax=1.5, ymin=-2.5, ymax=2.5,
    xtick={-2,0}, ytick={-2,0,2},
    tick label style={font=\tiny},
    axis line style={warmgrey},
    grid=major, grid style={warmgrey!12},
    axis background/.style={fill=sandstorm, fill opacity=0.3},
    title={\tiny Eigenvalues},
    every axis title/.style={font=\tiny},
]
    % Imaginary axis
    \addplot[warmgrey!30, thin] coordinates {(0, -2.5) (0, 2.5)};

    % Eigenvalues at -a +/- j*b
    \addplot[only marks, mark=*, mark size=2.5pt, atlantic]
        coordinates {(-\aval, \bval) (-\aval, -\bval)};

    % Label
    \node[font=\tiny, atlantic, anchor=west]
        at (axis cs:{-\aval+0.1}, {\bval})
        {$\lambda$};
\end{axis}

\end{tikzpicture}
\end{document}
