% Example 23: Magnetic Field of a Current Loop
% Cross-section of a current loop with Euler-integrated dipole field lines.
% Parameter: \PARAM (current I in amperes, range 0.5..5.0)
% Range: 0.5 to 5.0, recommended 30 frames
% Difficulty: intermediate
% Features demonstrated: magnetic field, Euler integration, heat map,
%   multi-layer wire cross-sections, current arrows, B-field profile

\documentclass[border=8pt,tikz]{standalone}
\usepackage{tikz}
\usepackage{amsmath}
\usetikzlibrary{arrows.meta, calc, decorations.markings}

% USC Brand Colors
\definecolor{garnet}{HTML}{73000A}
\definecolor{rose}{HTML}{CC2E40}
\definecolor{atlantic}{HTML}{466A9F}
\definecolor{congaree}{HTML}{1F414D}
\definecolor{horseshoe}{HTML}{65780B}
\definecolor{grass}{HTML}{CED318}
\definecolor{honeycomb}{HTML}{A49137}
\definecolor{warmgrey}{HTML}{676156}
\definecolor{sandstorm}{HTML}{FFF2E3}

\begin{document}
\begin{tikzpicture}[>=Stealth, every node/.style={font=\small}]

\pgfmathsetmacro{\Ival}{\PARAM}
\pgfmathsetmacro{\wireR}{1.0} % half-distance between wires

% Field strength background heat map (coarse 16x12 pixel grid)
\foreach \hx in {-3.6, -3.0, -2.4, -1.8, -1.2, -0.6, 0.0, 0.6, 1.2, 1.8, 2.4, 3.0, 3.6} {
    \foreach \hy in {-2.7, -2.1, -1.5, -0.9, -0.3, 0.3, 0.9, 1.5, 2.1, 2.7} {
        % B from right wire (out of page): Bx = -y/r^2, By = (x-1)/r^2
        \pgfmathsetmacro{\drx}{\hx - \wireR}
        \pgfmathsetmacro{\dry}{\hy}
        \pgfmathsetmacro{\rrR}{max(sqrt(\drx*\drx + \dry*\dry), 0.3)}
        % B from left wire (into page): opposite sign
        \pgfmathsetmacro{\dlx}{\hx + \wireR}
        \pgfmathsetmacro{\dly}{\hy}
        \pgfmathsetmacro{\rrL}{max(sqrt(\dlx*\dlx + \dly*\dly), 0.3)}
        % |B| proportional to I * (1/rR + 1/rL)
        \pgfmathsetmacro{\Bmag}{\Ival * (1.0/\rrR + 1.0/\rrL) * 0.3}
        \pgfmathtruncatemacro{\Bpct}{min(round(\Bmag * 15), 70)}
        \fill[atlantic!\Bpct!sandstorm, opacity=0.35]
            ({\hx-0.3}, {\hy-0.3}) rectangle ({\hx+0.3}, {\hy+0.3});
    }
}

% Grid
\draw[warmgrey!12, very thin, step=0.5] (-4.2,-3.2) grid (4.2,3.2);

% Euler-integrated field lines from right wire (dipole superposition)
\foreach \startAngle in {45, 90, 135, 225, 270, 315} {
    \pgfmathsetmacro{\startR}{0.25}
    \pgfmathsetmacro{\cx}{\wireR + \startR * cos(\startAngle)}
    \pgfmathsetmacro{\cy}{\startR * sin(\startAngle)}
    \pgfmathsetmacro{\stepLen}{0.12}
    % Direction: CCW around right wire (out of page)
    \pgfmathsetmacro{\prevx}{\cx}
    \pgfmathsetmacro{\prevy}{\cy}
    \foreach \step in {1, 2, ..., 55} {
        % B-field from right wire (out of page, CCW)
        \pgfmathsetmacro{\drx}{\cx - \wireR}
        \pgfmathsetmacro{\dry}{\cy}
        \pgfmathsetmacro{\rrR}{max(sqrt(\drx*\drx + \dry*\dry), 0.2)}
        \pgfmathsetmacro{\BxR}{-\dry / (\rrR*\rrR)}
        \pgfmathsetmacro{\ByR}{\drx / (\rrR*\rrR)}
        % B-field from left wire (into page, CW)
        \pgfmathsetmacro{\dlx}{\cx + \wireR}
        \pgfmathsetmacro{\dly}{\cy}
        \pgfmathsetmacro{\rrL}{max(sqrt(\dlx*\dlx + \dly*\dly), 0.2)}
        \pgfmathsetmacro{\BxL}{\dly / (\rrL*\rrL)}
        \pgfmathsetmacro{\ByL}{-\dlx / (\rrL*\rrL)}
        % Superposition
        \pgfmathsetmacro{\Bx}{\Ival * (\BxR + \BxL)}
        \pgfmathsetmacro{\By}{\Ival * (\ByR + \ByL)}
        \pgfmathsetmacro{\Bmag}{max(sqrt(\Bx*\Bx + \By*\By), 0.001)}
        \pgfmathsetmacro{\prevx}{\cx}
        \pgfmathsetmacro{\prevy}{\cy}
        \pgfmathsetmacro{\cx}{\cx + \Bx/\Bmag * \stepLen}
        \pgfmathsetmacro{\cy}{\cy + \By/\Bmag * \stepLen}
        % Color by distance (close=garnet, far=atlantic)
        \pgfmathsetmacro{\distCenter}{sqrt(\cx*\cx + \cy*\cy)}
        \pgfmathtruncatemacro{\gpct}{min(max(round(100 - \distCenter*25), 10), 90)}
        \pgfmathparse{(abs(\cx) < 4.0) && (abs(\cy) < 3.0) ? 1 : 0}
        \ifnum\pgfmathresult=1
            \draw[garnet!\gpct!atlantic, thin] (\prevx, \prevy) -- (\cx, \cy);
        \fi
    }
}

% Symmetry axis
\draw[dashed, warmgrey!40, thin] (0, -3.0) -- (0, 3.0)
    node[above, font=\tiny, warmgrey]{axis};

% Multi-layer wire cross-sections
% Right conductor (out of page)
\filldraw[black!60] (\wireR, 0) circle (8pt);       % insulation
\filldraw[honeycomb!70] (\wireR, 0) circle (5pt);   % copper
\filldraw[white] (\wireR, 0) circle (1.5pt);         % dot symbol
\node[font=\tiny, anchor=north, yshift=-10pt] at (\wireR, 0) {$\odot$};

% Left conductor (into page)
\filldraw[black!60] (-\wireR, 0) circle (8pt);      % insulation
\filldraw[honeycomb!70] (-\wireR, 0) circle (5pt);  % copper
\draw[white, thick] ({-\wireR-0.04}, -0.04) -- ({-\wireR+0.04}, 0.04);
\draw[white, thick] ({-\wireR-0.04}, 0.04) -- ({-\wireR+0.04}, -0.04);
\node[font=\tiny, anchor=north, yshift=-10pt] at (-\wireR, 0) {$\otimes$};

% Current flow arrows around wire perimeter
\pgfmathsetmacro{\arrowR}{0.38}
\foreach \ca in {30, 150, 270} {
    \draw[-{Stealth[length=2pt]}, atlantic, thick]
        ({\wireR + \arrowR*cos(\ca)}, {\arrowR*sin(\ca)})
        arc (\ca:{\ca+50}:\arrowR);
}
\foreach \ca in {30, 150, 270} {
    \draw[-{Stealth[length=2pt]}, atlantic, thick]
        ({-\wireR + \arrowR*cos(\ca+180)}, {\arrowR*sin(\ca+180)})
        arc ({\ca+180}:{\ca+230}:\arrowR);
}

% Info box
\node[anchor=north west, font=\normalsize, fill=white, fill opacity=0.9,
      text opacity=1, inner sep=4pt, draw=warmgrey!50, align=left]
    at (-4.1, 3.1)
    {$I = \pgfmathprintnumber[fixed, precision=1]{\Ival}$ A};

\end{tikzpicture}
\end{document}
