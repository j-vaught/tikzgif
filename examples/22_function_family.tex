% Example 22: Function Family Animation
% Animates y = x^n as the exponent sweeps from 0.3 to 5.0.
% Parameter: \PARAM (exponent n, range 0.3..5.0)
% Range: 0.3 to 5.0, recommended 48 frames
% Difficulty: intermediate
% Features demonstrated: power curves, function families, tangent lines,
%   area shading, derivative overlay, zoomed inset, convexity annotation
\documentclass[border=5pt]{standalone}
\usepackage{tikz}
\usepackage{pgfplots}
\usepackage{amsmath}
\pgfplotsset{compat=1.18}
\usepgfplotslibrary{fillbetween}

% USC Brand Colors
\definecolor{garnet}{HTML}{73000A}
\definecolor{rose}{HTML}{CC2E40}
\definecolor{atlantic}{HTML}{466A9F}
\definecolor{congaree}{HTML}{1F414D}
\definecolor{horseshoe}{HTML}{65780B}
\definecolor{grass}{HTML}{CED318}
\definecolor{honeycomb}{HTML}{A49137}
\definecolor{warmgrey}{HTML}{676156}
\definecolor{sandstorm}{HTML}{FFF2E3}

\begin{document}
\begin{tikzpicture}

\pgfmathsetmacro{\nval}{\PARAM}

% Compute tangent line at x=0.5: slope = n * 0.5^(n-1), y0 = 0.5^n
\pgfmathsetmacro{\ytan}{pow(0.5, \nval)}
\pgfmathsetmacro{\slopetan}{\nval * pow(0.5, \nval - 1)}
% Tangent line endpoints: y = y0 + slope*(x - 0.5), clamp x in [0.1, 0.9]
\pgfmathsetmacro{\ytanL}{\ytan + \slopetan*(0.15 - 0.5)}
\pgfmathsetmacro{\ytanR}{\ytan + \slopetan*(0.85 - 0.5)}

% Integral value: int_0^1 x^n dx = 1/(n+1)
\pgfmathsetmacro{\integralval}{1.0/(\nval + 1.0)}

% Convexity: n>1 convex, n<1 concave, n=1 linear
\pgfmathsetmacro{\isconvex}{\nval > 1.05 ? 1 : (\nval < 0.95 ? -1 : 0)}

\begin{axis}[
    name=mainax,
    width=10cm, height=8cm,
    xmin=-0.05, xmax=2.3, ymin=-0.15, ymax=5.5,
    xlabel={$x$}, ylabel={$y$},
    title={$y = x^{\pgfmathprintnumber[fixed, precision=2]{\nval}}$},
    grid=major, grid style={warmgrey!20}, thick,
    every axis title/.style={font=\large},
    tick label style={font=\small},
    clip=false,
]
    % Ghost family (faint reference curves)
    \foreach \ghostN in {0.5, 1.0, 2.0, 3.0, 4.0, 5.0} {
        \addplot[domain=0.01:2.2, samples=80, warmgrey!25, thin] {x^\ghostN};
    }

    % Shaded area under curve from 0 to 1
    \addplot[domain=0:1, samples=100, name path=funcpath, draw=none] {x^\nval};
    \addplot[domain=0:1, samples=2, name path=xaxis, draw=none] {0};
    \addplot[fill=atlantic, fill opacity=0.12] fill between[of=funcpath and xaxis];

    % Area annotation
    \node[font=\footnotesize, atlantic!80!black, anchor=north]
        at (axis cs:0.5, {0.5*pow(0.5,\nval)})
        {$A = \frac{1}{\pgfmathprintnumber[fixed,precision=2]{\nval}\!+\!1}
         = \pgfmathprintnumber[fixed, precision=3]{\integralval}$};

    % Derivative curve overlay: f'(x) = n * x^(n-1)
    \pgfmathsetmacro{\nvalm}{\nval - 1}
    \addplot[domain=0.01:2.2, samples=150, garnet, thin, dashed]
        {\nval * x^(\nvalm)};
    \node[font=\tiny, garnet, anchor=west]
        at (axis cs:2.22, {\nval * 2.2^(\nvalm)})
        {$f'$};

    % Current curve (main)
    \addplot[domain=0.01:2.2, samples=200, atlantic, ultra thick] {x^\nval};

    % Identity line
    \addplot[domain=0:2.2, samples=2, dashed, black!30, thin] {x};

    % Tangent line at x = 0.5
    \addplot[domain=0.15:0.85, samples=2, rose, thick] {\ytan + \slopetan*(x - 0.5)};
    \addplot[only marks, mark=*, mark size=2.5pt, rose]
        coordinates {(0.5, \ytan)};
    \node[font=\tiny, rose, anchor=south west]
        at (axis cs:0.52, \ytan)
        {$m = \pgfmathprintnumber[fixed, precision=2]{\slopetan}$};

    % Fixed point (1,1)
    \addplot[only marks, mark=*, mark size=3pt, garnet] coordinates {(1, 1)};
    \node[font=\footnotesize, anchor=south west, garnet]
        at (axis cs:1.05, 1.05) {$(1,\,1)$};

    % Convexity annotation
    \pgfmathparse{\isconvex > 0.5 ? "Convex" : (\isconvex < -0.5 ? "Concave" : "Linear")}
    \edef\convlabel{\pgfmathresult}
    \pgfmathparse{\isconvex > 0.5 ? "horseshoe" : (\isconvex < -0.5 ? "rose" : "warmgrey")}
    \edef\convcolor{\pgfmathresult}
    \node[font=\footnotesize\bfseries, \convcolor, anchor=north east,
          fill=white, fill opacity=0.8, text opacity=1, inner sep=2pt]
        at (axis cs:2.2, 5.3) {\convlabel};
\end{axis}

% Zoomed inset near (1,1)
\begin{axis}[
    at={(mainax.north east)}, anchor=north east,
    shift={(-0.4cm,-1.2cm)},
    width=3.2cm, height=3.2cm,
    xmin=0.75, xmax=1.25, ymin=0.75, ymax=1.25,
    xtick={0.8, 1.0, 1.2}, ytick={0.8, 1.0, 1.2},
    tick label style={font=\tiny},
    grid=major, grid style={warmgrey!15},
    axis line style={warmgrey},
    axis background/.style={fill=sandstorm, fill opacity=0.3},
    title={\tiny Inset near $(1,1)$},
    every axis title/.style={font=\tiny},
]
    % Ghost family in inset
    \foreach \ghostN in {0.5, 1.0, 2.0, 3.0, 5.0} {
        \addplot[domain=0.75:1.25, samples=60, warmgrey!20, thin] {x^\ghostN};
    }
    % Current curve in inset
    \addplot[domain=0.75:1.25, samples=100, atlantic, very thick] {x^\nval};
    % Identity
    \addplot[domain=0.75:1.25, samples=2, dashed, black!30, thin] {x};
    % Fixed point
    \addplot[only marks, mark=*, mark size=2pt, garnet] coordinates {(1, 1)};
\end{axis}

% Info box with exponent and integral
\node[anchor=north west, fill=white, fill opacity=0.85, text opacity=1,
      draw=warmgrey!50, inner sep=4pt, font=\footnotesize,
      align=left]
    at ([shift={(0.3cm,-0.2cm)}]mainax.north west)
    {$n = \pgfmathprintnumber[fixed, precision=2]{\nval}$\\[2pt]
     $\displaystyle\int_0^1\!x^n\,dx = \pgfmathprintnumber[fixed, precision=3]{\integralval}$};

\end{tikzpicture}
\end{document}
