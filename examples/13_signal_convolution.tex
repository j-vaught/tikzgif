% Example 13: Signal Convolution
% Sliding window convolution of a rectangular pulse with a triangular pulse.
% Shows f(tau), g(t-tau), and the resulting convolution integral.
% Parameter: \PARAM (time shift t, -2 to 4)
% Range: -2 to 4, recommended 60 frames
% Difficulty: advanced
% Features demonstrated: pgfplots, filled regions, multiple synchronized plots
\documentclass[tikz]{standalone}
\usepackage{pgfplots}
\usepackage{xcolor}
\pgfplotsset{compat=1.18}
\usepgfplotslibrary{fillbetween}

% USC Brand Colors
\definecolor{garnet}{HTML}{73000A}
\definecolor{rose}{HTML}{CC2E40}
\definecolor{atlantic}{HTML}{466A9F}
\definecolor{congaree}{HTML}{1F414D}
\definecolor{horseshoe}{HTML}{65780B}
\definecolor{grass}{HTML}{CED318}
\definecolor{honeycomb}{HTML}{A49137}
\definecolor{warmgrey}{HTML}{676156}
\definecolor{sandstorm}{HTML}{FFF2E3}

\begin{document}
\begin{tikzpicture}
  \pgfmathsetmacro{\tshift}{\PARAM}

  % --- Top plot: f(tau) and g(t - tau) ---
  \begin{axis}[
    name=topplot,
    at={(0,4.5cm)},
    width=10cm, height=4.5cm,
    domain=-3:5,
    samples=200,
    axis lines=middle,
    xlabel={$\tau$},
    ylabel={},
    ymin=-0.3, ymax=1.5,
    xmin=-3, xmax=5,
    grid=major,
    grid style={warmgrey!20},
    title={\small Signals: $f(\tau)$ and $g(t - \tau)$},
    every axis title/.style={at={(0.5,1.1)}, font=\small},
    legend pos=north east,
    legend style={font=\tiny},
    clip=false,
  ]
    % f(tau): rectangular pulse from 0 to 2, height 1
    \addplot[name path=f, garnet, very thick]
      coordinates {(-3,0) (0,0) (0,1) (2,1) (2,0) (5,0)};
    \addlegendentry{$f(\tau)$}

    % g(t - tau): triangular pulse, peak at t-0.5, base from t-1 to t
    % g(x) = triangle on [-1, 1], peak=1 at x=0
    % g(t - tau) = g evaluated at (t - tau), i.e., triangle centered at t
    \addplot[atlantic, very thick]
      coordinates {
        (-3, 0)
        ({\tshift - 1}, 0)
        ({\tshift}, 1)
        ({\tshift + 1}, 0)
        (5, 0)
      };
    \addlegendentry{$g(t - \tau)$}

    % Shade overlap region
    \addplot[name path=g, atlantic, thick, draw=none]
      coordinates {
        (-3, 0)
        ({\tshift - 1}, 0)
        ({\tshift}, 1)
        ({\tshift + 1}, 0)
        (5, 0)
      };
    \addplot[honeycomb!50, opacity=0.3] fill between[of=f and g,
      soft clip={domain=-3:5}];
  \end{axis}

  % --- Bottom plot: convolution result (f * g)(t) ---
  \begin{axis}[
    at={(0,0)},
    width=10cm, height=4.5cm,
    domain=-3:5,
    samples=300,
    axis lines=middle,
    xlabel={$t$},
    ylabel={$(f * g)(t)$},
    ymin=-0.3, ymax=1.5,
    xmin=-3, xmax=5,
    grid=major,
    grid style={warmgrey!20},
    title={\small Convolution: $(f * g)(t)$},
    every axis title/.style={at={(0.5,1.1)}, font=\small},
    clip=false,
  ]
    % Analytical convolution of rect[0,2] * triangle[-1,1]:
    % Piecewise function computed analytically:
    % t in [-1, 0]:  0.5*(t+1)^2
    % t in [0, 2]:   -0.5*t^2 + t + 0.5  (peak of 1 at t=1)
    % t in [2, 3]:   0.5*(3-t)^2
    \addplot[warmgrey!60, very thick, smooth] {
      (x >= -1 && x < 0) * (0.5*(x+1)^2) +
      (x >= 0  && x < 2) * (-0.5*x*x + x + 0.5) +
      (x >= 2  && x < 3) * (0.5*(3 - x)^2)
    };

    % Traced portion up to current t
    \addplot[garnet, very thick, smooth, domain=-3:\tshift] {
      (x >= -1 && x < 0) * (0.5*(x+1)^2) +
      (x >= 0  && x < 2) * (-0.5*x*x + x + 0.5) +
      (x >= 2  && x < 3) * (0.5*(3 - x)^2)
    };

    % Current point marker
    \pgfmathsetmacro{\convval}{%
      (\tshift >= -1 && \tshift < 0) * (0.5*(\tshift+1)^2) +
      (\tshift >= 0  && \tshift < 2) * (-0.5*\tshift*\tshift + \tshift + 0.5) +
      (\tshift >= 2  && \tshift < 3) * (0.5*(3 - \tshift)^2)
    }
    \addplot[only marks, mark=*, mark size=3pt, garnet]
      coordinates {(\tshift, \convval)};

    % Vertical dashed line at current t
    \draw[dashed, garnet!50, thin] (axis cs:\tshift, 0) -- (axis cs:\tshift, \convval);

    \node[anchor=south west, font=\small] at (axis cs:-2.8, 1.2)
      {$t = \pgfmathprintnumber[fixed, precision=2]{\tshift}$};
  \end{axis}
\end{tikzpicture}
\end{document}
