% Example 13d: Signal Convolution (Exponential * Step)
% Sliding window convolution of a decaying exponential with a unit step.
% Shows f(tau), g(t-tau), and the resulting convolution integral.
% Parameter: \PARAM (time shift t, -2 to 6)
% Range: -2 to 6, recommended 60 frames
% Difficulty: advanced
% Features demonstrated: pgfplots, filled regions, multiple synchronized plots
\documentclass[tikz]{standalone}
\usepackage{pgfplots}
\usepackage{xcolor}
\pgfplotsset{compat=1.18}

% USC Brand Colors
\definecolor{garnet}{HTML}{73000A}
\definecolor{rose}{HTML}{CC2E40}
\definecolor{atlantic}{HTML}{466A9F}
\definecolor{congaree}{HTML}{1F414D}
\definecolor{horseshoe}{HTML}{65780B}
\definecolor{grass}{HTML}{CED318}
\definecolor{honeycomb}{HTML}{A49137}
\definecolor{warmgrey}{HTML}{676156}
\definecolor{sandstorm}{HTML}{FFF2E3}

\begin{document}
\begin{tikzpicture}
  \useasboundingbox (-2bp, -2bp) rectangle (244bp, 234bp);
  \pgfmathsetmacro{\tshift}{\PARAM}

  % --- Top plot: f(tau) and g(t - tau) ---
  \begin{axis}[
    name=topplot,
    at={(0,4.5cm)},
    width=10cm, height=4.5cm,
    domain=-3:7,
    samples=300,
    axis lines=middle,
    xlabel={$\tau$},
    ylabel={},
    ymin=-0.3, ymax=1.5,
    xmin=-3, xmax=7,
    grid=major,
    grid style={warmgrey!20},
    title={\small Signals: $f(\tau)$ and $g(t - \tau)$},
    every axis title/.style={at={(0.5,1.1)}, font=\small},
    legend pos=north east,
    legend style={font=\tiny},
    clip=false,
  ]
    % f(tau) = e^{-tau} u(tau): decaying exponential for tau >= 0
    \addplot[garnet, very thick] {
      (x >= 0) * exp(-x)
    };
    \addlegendentry{$f(\tau) = e^{-\tau}u(\tau)$}

    % g(t - tau) = u(t - tau): step function, equals 1 when tau <= t
    \addplot[atlantic, very thick] {
      (x <= \tshift) * 1
    };
    \addlegendentry{$g(t - \tau) = u(t - \tau)$}

    % Shade overlap region: where both f(tau) > 0 and g(t-tau) > 0
    % That is: tau >= 0 AND tau <= t, i.e., tau in [0, max(0, t)]
    \pgfmathsetmacro{\clipRight}{max(0, \tshift)}
    \pgfmathparse{\clipRight > 0 ? 1 : 0}
    \ifnum\pgfmathresult=1
      \addplot[fill=honeycomb!50, opacity=0.3, draw=none,
        domain=0:\clipRight, samples=100]
        {exp(-x)} \closedcycle;
    \fi
  \end{axis}

  % --- Bottom plot: convolution result (f * g)(t) ---
  \begin{axis}[
    at={(0,0)},
    width=10cm, height=4.5cm,
    domain=-3:7,
    samples=300,
    axis lines=middle,
    xlabel={$t$},
    ylabel={$(f * g)(t)$},
    ymin=-0.3, ymax=1.5,
    xmin=-3, xmax=7,
    grid=major,
    grid style={warmgrey!20},
    title={\small Convolution: $(f * g)(t)$},
    every axis title/.style={at={(0.5,1.1)}, font=\small},
    clip=false,
  ]
    % Analytical convolution of e^{-t}u(t) * u(t):
    % t < 0: 0
    % t >= 0: 1 - e^{-t}
    \addplot[warmgrey!60, very thick, smooth] {
      (x >= 0) * (1 - exp(-x))
    };

    % Traced portion up to current t
    \addplot[garnet, very thick, smooth, domain=-3:\tshift] {
      (x >= 0) * (1 - exp(-x))
    };

    % Current point marker
    \pgfmathsetmacro{\convval}{%
      (\tshift >= 0) * (1 - exp(-\tshift))
    }
    \addplot[only marks, mark=*, mark size=3pt, garnet]
      coordinates {(\tshift, \convval)};

    % Vertical dashed line at current t
    \draw[dashed, garnet!50, thin] (axis cs:\tshift, 0) -- (axis cs:\tshift, \convval);

    % Asymptote dashed line at y = 1
    \draw[dashed, warmgrey!40, thin] (axis cs:-3, 1) -- (axis cs:7, 1);

    \node[anchor=south west, font=\small] at (axis cs:-2.8, 1.2)
      {$t = \pgfmathprintnumber[fixed, precision=2]{\tshift}$};
  \end{axis}
\end{tikzpicture}
\end{document}
