%%--- TIKZGIF META ---
%% name: em.magnetic_field_loop
%% title: Magnetic Field of a Current Loop
%% description: >
%%   Cross-section of a current loop with field lines scaling with current I.
%% author: J.C. Vaught
%% version: 1.0.0
%% domain: electromagnetics
%% tags: [EM, magnetic field, current loop, dipole]
%% engine: pdflatex
%% tikz_libraries: [arrows.meta, calc, decorations.markings]
%% latex_packages: [tikz, amsmath]
%% params:
%%   - name: I
%%     type: float
%%     default: 2.0
%%     min: 0.5
%%     max: 5.0
%%     step: 0.15
%%     sweep: true
%%     description: Current magnitude in amperes
%%     unit: "A"
%% fps: 12
%% frames: 30
%% loop: true
%% bounce: true
%%--- END META ---
\documentclass[border=8pt,tikz]{standalone}
\usepackage{tikz}
\usepackage{amsmath}
\usetikzlibrary{arrows.meta, calc, decorations.markings}

% USC Brand Colors
\definecolor{garnet}{HTML}{73000A}
\definecolor{rose}{HTML}{CC2E40}
\definecolor{atlantic}{HTML}{466A9F}
\definecolor{congaree}{HTML}{1F414D}
\definecolor{horseshoe}{HTML}{65780B}
\definecolor{grass}{HTML}{CED318}
\definecolor{honeycomb}{HTML}{A49137}
\definecolor{warmgrey}{HTML}{676156}
\definecolor{sandstorm}{HTML}{FFF2E3}

\begin{document}
\begin{tikzpicture}[>=Stealth, every node/.style={font=\small}]

\pgfmathsetmacro{\Ival}{{{{ I }}}}
\pgfmathsetmacro{\lineScale}{0.25 + 0.15 * \Ival}
\pgfmathsetmacro{\nLines}{min(floor(\Ival * 2 + 1), 10)}

% Grid
\draw[warmgrey!15, very thin, step=0.5] (-4.5,-3.5) grid (4.5,3.5);

% Right conductor (out of page)
\filldraw[black!80] (1, 0) circle (6pt);
\filldraw[white] (1, 0) circle (2pt);
\node[below=5mm, font=\footnotesize] at (1, 0) {$\odot\; I$};

% Left conductor (into page)
\filldraw[black!80] (-1, 0) circle (6pt);
\draw[white, thick] (-1.06, -0.06) -- (-0.94, 0.06);
\draw[white, thick] (-1.06, 0.06) -- (-0.94, -0.06);
\node[below=5mm, font=\footnotesize] at (-1, 0) {$\otimes\; I$};

% Local field lines around right conductor
\foreach \k in {1, 2, ..., 6} {
    \pgfmathsetmacro{\rr}{0.25 + \k * 0.2 * \lineScale}
    \pgfmathparse{\k <= \nLines ? 1 : 0}
    \ifnum\pgfmathresult=1
        \draw[atlantic, thin,
              decoration={markings,
                  mark=at position 0.25 with {\arrow{Stealth[length=2.5pt]}},
                  mark=at position 0.75 with {\arrow{Stealth[length=2.5pt]}}},
              postaction={decorate}]
            (1, 0) circle [x radius=\rr, y radius=\rr*0.9];
    \fi
}

% Local field lines around left conductor
\foreach \k in {1, 2, ..., 6} {
    \pgfmathsetmacro{\rr}{0.25 + \k * 0.2 * \lineScale}
    \pgfmathparse{\k <= \nLines ? 1 : 0}
    \ifnum\pgfmathresult=1
        \draw[atlantic, thin,
              decoration={markings,
                  mark=at position 0.25 with {\arrowreversed{Stealth[length=2.5pt]}},
                  mark=at position 0.75 with {\arrowreversed{Stealth[length=2.5pt]}}},
              postaction={decorate}]
            (-1, 0) circle [x radius=\rr, y radius=\rr*0.9];
    \fi
}

% Dipole field lines (encircling both)
\foreach \k in {1, 2, ..., 5} {
    \pgfmathsetmacro{\rx}{1.5 + \k * 0.45 * \lineScale}
    \pgfmathsetmacro{\ry}{0.8 + \k * 0.4 * \lineScale}
    \pgfmathparse{\k <= \nLines ? 1 : 0}
    \ifnum\pgfmathresult=1
        \draw[garnet, thin,
              decoration={markings,
                  mark=at position 0.0 with {\arrow{Stealth[length=2.5pt]}},
                  mark=at position 0.5 with {\arrow{Stealth[length=2.5pt]}}},
              postaction={decorate}]
            (0, 0) ellipse [x radius=\rx, y radius=\ry];
    \fi
}

\node[anchor=north west, font=\normalsize, fill=white, inner sep=3pt, draw=warmgrey]
    at (-4.4, 3.4)
    {$I = \pgfmathprintnumber[fixed, precision=1]{\Ival}$ A};

\draw[dashed, warmgrey!50, thin] (0, -3.3) -- (0, 3.3)
    node[above, font=\tiny, gray]{axis};

\end{tikzpicture}
\end{document}
