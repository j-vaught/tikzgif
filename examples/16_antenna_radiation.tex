% Example 16: Dual Antenna Radiation Patterns vs Frequency
% Side-by-side microstrip patch and dipole antennas with polar radiation
% patterns that evolve as frequency sweeps from f/f0 = 1.0 to 3.0.
% Parameter: \PARAM (frequency ratio f/f0, range 1.0..3.0)
% Range: 1.0 to 3.0, recommended 60 frames at 15 fps
% Difficulty: advanced
% Features demonstrated: dual polar plots, antenna schematics, frequency sweep,
%   narrowband vs broadband behavior, ghost reference patterns, dB-scale grid

\documentclass[border=5pt,tikz]{standalone}
\usepackage{tikz}
\usepackage{amsmath}
\usetikzlibrary{arrows.meta,patterns,decorations.markings}

% USC Brand Colors
\definecolor{garnet}{HTML}{73000A}
\definecolor{rose}{HTML}{CC2E40}
\definecolor{atlantic}{HTML}{466A9F}
\definecolor{congaree}{HTML}{1F414D}
\definecolor{horseshoe}{HTML}{65780B}
\definecolor{grass}{HTML}{CED318}
\definecolor{honeycomb}{HTML}{A49137}
\definecolor{warmgrey}{HTML}{676156}
\definecolor{sandstorm}{HTML}{FFF2E3}

\begin{document}
\begin{tikzpicture}[>=Stealth, every node/.style={font=\small}]

% Fix bounding box so tikzgif does not crop
\useasboundingbox (-9,-8) rectangle (9,6);
\clip (-9,-8) rectangle (9,6);

% ============================================================
% PARAMETER SETUP
% ============================================================
\pgfmathsetmacro{\freqRatio}{\PARAM}
\pgfmathsetmacro{\Rmax}{2.8}
\pgfmathsetmacro{\Qfactor}{10}

% Patch efficiency: eta = 1 / (1 + Q^2*(f/f0 - f0/f)^2)
\pgfmathsetmacro{\mismatchArg}{\freqRatio - 1.0/\freqRatio}
\pgfmathsetmacro{\etaVal}{1.0 / (1.0 + \Qfactor*\Qfactor*\mismatchArg*\mismatchArg)}
\pgfmathsetmacro{\sqrtEta}{sqrt(\etaVal)}
\pgfmathsetmacro{\etaPct}{\etaVal * 100}

% Dipole electrical length in wavelengths
\pgfmathsetmacro{\elecLen}{0.5 * \freqRatio}

% uDeg for dipole: 90 * freqRatio (= pi/2 * f/f0 in radians, but pgfmath uses degrees)
\pgfmathsetmacro{\uDeg}{90.0 * \freqRatio}

% patchArg for patch: 90 * freqRatio
\pgfmathsetmacro{\patchArg}{90.0 * \freqRatio}

% dB-scale grid ring radii
\pgfmathsetmacro{\RdBzero}{\Rmax}
\pgfmathsetmacro{\RdBthree}{\Rmax * 0.7079}
\pgfmathsetmacro{\RdBten}{\Rmax * 0.3162}
\pgfmathsetmacro{\RdBtwenty}{\Rmax * 0.1}

% ============================================================
% COORDINATE CONVENTION
% ============================================================
% TikZ polar angle \t: 0=right, 90=up, 180=left, 270=down
% Physical angle theta_phys (from dipole/vertical axis):
%   theta_phys = 90 - \t, so cos(theta_phys)=sin(\t), sin(theta_phys)=cos(\t)
%
% DIPOLE (vertical axis = up):
%   E = [cos(u*cos(theta_phys)) - cos(u)] / sin(theta_phys)
%     = [cos(u*sin(\t)) - cos(u)] / cos(\t)
%   Nulls at \t=90,270 (along dipole axis). Peaks at \t=0,180 (broadside).
%
% PATCH (radiates upward = broadside):
%   E = sqrt(eta) * |cos(patchArg * sin(theta_phys))| * |cos(theta_phys)|
%     = sqrt(eta) * |cos(patchArg * cos(\t))| * |sin(\t)|
%   Peaks at \t=90 (up). Zero in lower hemisphere (\t in 181..359).

% ============================================================
% DIPOLE NORMALIZATION - sample at multiple TikZ angles to find peak
% ============================================================
% Sampling at \t = 0 (broadside), 15, 30, 36, 45, 60, 75 degrees
% where cos(\t) is safely > 0
\pgfmathsetmacro{\dipSampA}{abs(cos(0)) > 0.1 ? abs((cos(\uDeg * sin(0)) - cos(\uDeg)) / cos(0)) : 0}
\pgfmathsetmacro{\dipSampB}{abs(cos(15)) > 0.1 ? abs((cos(\uDeg * sin(15)) - cos(\uDeg)) / cos(15)) : 0}
\pgfmathsetmacro{\dipSampC}{abs(cos(30)) > 0.1 ? abs((cos(\uDeg * sin(30)) - cos(\uDeg)) / cos(30)) : 0}
\pgfmathsetmacro{\dipSampD}{abs(cos(36)) > 0.1 ? abs((cos(\uDeg * sin(36)) - cos(\uDeg)) / cos(36)) : 0}
\pgfmathsetmacro{\dipSampE}{abs(cos(45)) > 0.1 ? abs((cos(\uDeg * sin(45)) - cos(\uDeg)) / cos(45)) : 0}
\pgfmathsetmacro{\dipSampF}{abs(cos(60)) > 0.1 ? abs((cos(\uDeg * sin(60)) - cos(\uDeg)) / cos(60)) : 0}
\pgfmathsetmacro{\dipSampG}{abs(cos(75)) > 0.1 ? abs((cos(\uDeg * sin(75)) - cos(\uDeg)) / cos(75)) : 0}
\pgfmathsetmacro{\dipNorm}{max(max(max(\dipSampA,\dipSampB),max(\dipSampC,\dipSampD)),max(max(\dipSampE,\dipSampF),\dipSampG))}
\pgfmathsetmacro{\dipNorm}{max(\dipNorm, 0.001)}

% ============================================================
% TITLE
% ============================================================
\node[font=\large\bfseries, anchor=north] at (0, 5.8)
    {Antenna Radiation Patterns vs.\ Frequency};

% ============================================================
% PANEL LABELS
% ============================================================
\node[font=\normalsize\bfseries, garnet, anchor=south] at (-4.5, 5.0)
    {Microstrip Patch};
\node[font=\normalsize\bfseries, atlantic, anchor=south] at (4.5, 5.0)
    {Dipole};

% ============================================================
% PATCH ANTENNA DRAWING (left, centered at x=-4.5, y~2.5..4.5)
% ============================================================
\begin{scope}[shift={(-4.5, 3.2)}]
    % Ground plane
    \fill[warmgrey!50] (-2.2, -0.9) rectangle (2.2, -0.7);
    \draw[warmgrey!70, thick] (-2.2, -0.7) -- (2.2, -0.7);
    % Hatching on ground
    \foreach \gx in {-2.0,-1.6,...,2.0} {
        \draw[warmgrey!40, thin] (\gx, -0.9) -- (\gx+0.2, -0.7);
    }
    \node[font=\tiny, warmgrey, anchor=west] at (2.3, -0.8) {Ground};

    % Substrate
    \fill[sandstorm] (-1.6, -0.7) rectangle (1.6, -0.2);
    \draw[warmgrey!50, thin] (-1.6, -0.7) rectangle (1.6, -0.2);
    \node[font=\tiny, warmgrey, anchor=west] at (1.7, -0.45) {Substrate};

    % Patch element
    \fill[garnet!25] (-1.2, -0.2) rectangle (1.2, 0.1);
    \draw[garnet, thick] (-1.2, -0.2) rectangle (1.2, 0.1);
    \node[font=\tiny, garnet, anchor=west] at (1.3, -0.05) {Patch};

    % Feed probe
    \draw[black!70, thick] (0, -0.9) -- (0, -0.2);
    \filldraw[black!70] (0, -0.2) circle (1.2pt);

    % E-field arrows from patch edges (opacity scaled by sqrtEta)
    \foreach \ey in {-0.15, -0.05, 0.05} {
        \draw[rose, thick, ->, opacity=\sqrtEta] (-1.2, \ey) -- (-1.7, \ey+0.15);
        \draw[rose, thick, ->, opacity=\sqrtEta] (1.2, \ey) -- (1.7, \ey+0.15);
    }
\end{scope}

% ============================================================
% DIPOLE ANTENNA DRAWING (right, centered at x=4.5, y~2.5..4.5)
% ============================================================
\begin{scope}[shift={(4.5, 3.2)}]
    % Upper arm
    \draw[atlantic, line width=3pt, line cap=round] (0, 0.15) -- (0, 1.3);
    % Lower arm
    \draw[atlantic, line width=3pt, line cap=round] (0, -0.15) -- (0, -1.3);

    % Feed gap
    \filldraw[atlantic] (0, 0.15) circle (2pt);
    \filldraw[atlantic] (0, -0.15) circle (2pt);
    % Source symbol
    \node[font=\small, atlantic] at (0.35, 0) {$\sim$};

    % Current distribution (sinusoidal standing wave)
    \pgfmathsetmacro{\halfSins}{\freqRatio}
    \draw[congaree!40, thick, samples=80, domain=-1.3:1.3, variable=\yy]
        plot ({0.4 * sin(180 * \halfSins * (\yy + 1.3) / 2.6)}, {\yy});

    % Length label
    \draw[black!60, thin, |->|] (-0.7, -1.3) -- (-0.7, 1.3);
    \node[font=\tiny, black!60, anchor=east, align=right] at (-0.8, 0)
        {$L = 0.5\lambda_0$};

    % Electrical length (updates with frequency)
    \node[font=\tiny, congaree, anchor=west] at (0.7, -0.8)
        {$= \pgfmathprintnumber[fixed, precision=1]{\elecLen}\lambda$};
\end{scope}

% ============================================================
% PATCH POLAR PLOT (left panel, centered at (-4.5, -3.0))
% ============================================================
\begin{scope}[shift={(-4.5, -3.0)}]
    % dB grid rings
    \draw[warmgrey!30, thin] (0,0) circle (\RdBzero);
    \draw[warmgrey!30, thin] (0,0) circle (\RdBthree);
    \draw[warmgrey!25, thin] (0,0) circle (\RdBten);
    \draw[warmgrey!20, thin] (0,0) circle (\RdBtwenty);

    % dB labels
    \node[font=\tiny, warmgrey!50, anchor=south west] at ({\RdBzero*0.707}, {\RdBzero*0.707}) {0\,dB};
    \node[font=\tiny, warmgrey!50, anchor=south west] at ({\RdBthree*0.707}, {\RdBthree*0.707}) {$-3$};
    \node[font=\tiny, warmgrey!40, anchor=south west] at ({\RdBten*0.707}, {\RdBten*0.707}) {$-10$};

    % Angle markers
    \foreach \a in {0, 30, 60, ..., 330} {
        \draw[warmgrey!20, thin] (0,0) -- (\a:\Rmax+0.15);
        \node[font=\tiny, warmgrey!40] at (\a:\Rmax+0.4) {$\a^\circ$};
    }

    % Ghost reference pattern (f/f0=1, patch at resonance)
    % E = |cos(90*cos(\t))| * |sin(\t)|, upper hemisphere only (\t in 1..179)
    \fill[warmgrey!10, opacity=0.4, samples=360, domain=0:360, variable=\t]
        plot ({\t}: {
            (\t > 1 && \t < 179) ?
                \Rmax * abs(cos(90 * cos(\t))) * abs(sin(\t))
            : 0
        }) -- cycle;

    % Main patch pattern
    % E = sqrtEta * |cos(patchArg*cos(\t))| * |sin(\t)|, upper hemisphere only
    \fill[garnet!15, opacity=0.4, samples=360, domain=0:360, variable=\t]
        plot ({\t}: {
            (\t > 1 && \t < 179) ?
                \Rmax * \sqrtEta * abs(cos(\patchArg * cos(\t))) * abs(sin(\t))
            : 0
        }) -- cycle;

    \draw[garnet, very thick, samples=360, domain=0:360, variable=\t]
        plot ({\t}: {
            (\t > 1 && \t < 179) ?
                \Rmax * \sqrtEta * abs(cos(\patchArg * cos(\t))) * abs(sin(\t))
            : 0
        });

    % Efficiency label
    \node[anchor=north, font=\tiny, fill=white, fill opacity=0.8, text opacity=1,
          inner sep=2pt, draw=warmgrey!30] at (0, -\Rmax-0.3)
        {$\eta = \pgfmathprintnumber[fixed,precision=0]{\etaPct}\%$};
\end{scope}

% ============================================================
% DIPOLE POLAR PLOT (right panel, centered at (4.5, -3.0))
% ============================================================
\begin{scope}[shift={(4.5, -3.0)}]
    % dB grid rings
    \draw[warmgrey!30, thin] (0,0) circle (\RdBzero);
    \draw[warmgrey!30, thin] (0,0) circle (\RdBthree);
    \draw[warmgrey!25, thin] (0,0) circle (\RdBten);
    \draw[warmgrey!20, thin] (0,0) circle (\RdBtwenty);

    % dB labels
    \node[font=\tiny, warmgrey!50, anchor=south west] at ({\RdBzero*0.707}, {\RdBzero*0.707}) {0\,dB};
    \node[font=\tiny, warmgrey!50, anchor=south west] at ({\RdBthree*0.707}, {\RdBthree*0.707}) {$-3$};
    \node[font=\tiny, warmgrey!40, anchor=south west] at ({\RdBten*0.707}, {\RdBten*0.707}) {$-10$};

    % Angle markers
    \foreach \a in {0, 30, 60, ..., 330} {
        \draw[warmgrey!20, thin] (0,0) -- (\a:\Rmax+0.15);
        \node[font=\tiny, warmgrey!40] at (\a:\Rmax+0.4) {$\a^\circ$};
    }

    % Ghost reference pattern (f/f0=1, half-wave dipole)
    % E = |cos(90*sin(\t)) / cos(\t)|, with denominator clamped
    % At f/f0=1: cos(u)=cos(90)=0, so E = |cos(90*sin(\t))| / |cos(\t)|
    \fill[warmgrey!10, opacity=0.4, samples=360, domain=0:360, variable=\t]
        plot ({\t}: {
            abs(cos(\t)) > 0.08 ?
                \Rmax * abs(cos(90 * sin(\t))) / max(abs(cos(\t)), 0.08)
            : 0
        }) -- cycle;

    % Main dipole pattern
    % E = |[cos(u*sin(\t)) - cos(u)] / cos(\t)| / dipNorm
    \fill[atlantic!15, opacity=0.4, samples=360, domain=0:360, variable=\t]
        plot ({\t}: {
            abs(cos(\t)) > 0.08 ?
                min(\Rmax * abs(cos(\uDeg * sin(\t)) - cos(\uDeg)) / max(abs(cos(\t)), 0.08) / \dipNorm, \Rmax)
            : 0
        }) -- cycle;

    \draw[atlantic, very thick, samples=360, domain=0:360, variable=\t]
        plot ({\t}: {
            abs(cos(\t)) > 0.08 ?
                min(\Rmax * abs(cos(\uDeg * sin(\t)) - cos(\uDeg)) / max(abs(cos(\t)), 0.08) / \dipNorm, \Rmax)
            : 0
        });

    % Electrical length label
    \node[anchor=north, font=\tiny, fill=white, fill opacity=0.8, text opacity=1,
          inner sep=2pt, draw=warmgrey!30] at (0, -\Rmax-0.3)
        {$L = \pgfmathprintnumber[fixed,precision=1]{\elecLen}\lambda$};
\end{scope}

% ============================================================
% FREQUENCY READOUT AND PROGRESS BAR
% ============================================================
% Frequency label
\node[font=\normalsize, anchor=north] at (0, -7.0)
    {$f\,/\,f_0 = \pgfmathprintnumber[fixed,fixed zerofill,precision=2]{\freqRatio}$};

% Progress bar background
\fill[warmgrey!20] (-3, -7.55) rectangle (3, -7.35);
% Progress bar fill
\pgfmathsetmacro{\barFill}{(\freqRatio - 1.0) / (3.0 - 1.0)}
\fill[honeycomb] (-3, -7.55) rectangle ({-3 + 6*\barFill}, -7.35);
% Progress bar border
\draw[warmgrey!50, thin] (-3, -7.55) rectangle (3, -7.35);

\end{tikzpicture}
\end{document}
