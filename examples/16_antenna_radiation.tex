% Example 16: Antenna Radiation Pattern
% Polar radiation pattern f(theta) = |sin(theta)|^n with variable directivity.
% Parameter: \PARAM (radiation pattern exponent, range 1.0..8.0)
% Range: 1.0 to 8.0, recommended 30 frames
% Difficulty: intermediate
% Features demonstrated: polar plots, dB-scale grid, half-power beamwidth,
%   ghost reference pattern, directivity readout, sidelobe fill distinction

\documentclass[border=10pt,tikz]{standalone}
\usepackage{tikz}
\usepackage{amsmath}
\usetikzlibrary{arrows.meta}

% USC Brand Colors
\definecolor{garnet}{HTML}{73000A}
\definecolor{rose}{HTML}{CC2E40}
\definecolor{atlantic}{HTML}{466A9F}
\definecolor{congaree}{HTML}{1F414D}
\definecolor{horseshoe}{HTML}{65780B}
\definecolor{grass}{HTML}{CED318}
\definecolor{honeycomb}{HTML}{A49137}
\definecolor{warmgrey}{HTML}{676156}
\definecolor{sandstorm}{HTML}{FFF2E3}

\begin{document}
\begin{tikzpicture}[>=Stealth, every node/.style={font=\small}]

\pgfmathsetmacro{\nExp}{\PARAM}
\pgfmathsetmacro{\Rmax}{3.5}

% dB-scale concentric grid rings: -3dB, -10dB, -20dB, 0dB
% Map: R = Rmax * 10^(dB/20)
\pgfmathsetmacro{\RdBzero}{\Rmax}
\pgfmathsetmacro{\RdBthree}{\Rmax * 0.7079}  % 10^(-3/20)
\pgfmathsetmacro{\RdBten}{\Rmax * 0.3162}    % 10^(-10/20)
\pgfmathsetmacro{\RdBtwenty}{\Rmax * 0.1}    % 10^(-20/20)

% Draw dB rings
\draw[warmgrey!30, thin] (0,0) circle (\RdBzero);
\node[font=\tiny, warmgrey!50, anchor=south west] at ({\RdBzero*0.707}, {\RdBzero*0.707}) {0\,dB};
\draw[warmgrey!30, thin] (0,0) circle (\RdBthree);
\node[font=\tiny, warmgrey!50, anchor=south west] at ({\RdBthree*0.707}, {\RdBthree*0.707}) {-3};
\draw[warmgrey!25, thin] (0,0) circle (\RdBten);
\node[font=\tiny, warmgrey!40, anchor=south west] at ({\RdBten*0.707}, {\RdBten*0.707}) {-10};
\draw[warmgrey!20, thin] (0,0) circle (\RdBtwenty);
\node[font=\tiny, warmgrey!30, anchor=south west] at ({\RdBtwenty*0.707}, {\RdBtwenty*0.707}) {-20};

% Angle markers
\foreach \a in {0, 30, 60, ..., 330} {
    \draw[warmgrey!20, thin] (0,0) -- (\a:\Rmax+0.2);
    \node[font=\tiny, warmgrey!50] at (\a:\Rmax+0.5) {$\a^\circ$};
}

% Ghost reference pattern (n=1 dipole, faint)
\fill[warmgrey!10, opacity=0.4, samples=360, domain=0:360, variable=\t]
    plot ({\t}: {\Rmax * abs(sin(\t))}) -- cycle;

% Main pattern fill — main lobe in garnet, sidelobes in atlantic
% For sin^n, when n>=2 sidelobes are smaller; use single color for simplicity
% Main lobe fill
\fill[garnet!15, opacity=0.4, samples=360, domain=0:360, variable=\t]
    plot ({\t}: {\Rmax * (abs(sin(\t)) > 0.001 ? pow(abs(sin(\t)), \nExp) : 0)})
    -- cycle;

% Main pattern outline
\draw[garnet, very thick, samples=360, domain=0:360, variable=\t]
    plot ({\t}: {\Rmax * (abs(sin(\t)) > 0.001 ? pow(abs(sin(\t)), \nExp) : 0)});

% Half-power beamwidth arc
% theta_3dB = asin(0.5^(1/n)) in degrees
\pgfmathsetmacro{\halfPowerAngle}{asin(pow(0.5, 1.0/\nExp))}
\pgfmathsetmacro{\thetaUpper}{90 - \halfPowerAngle}
\pgfmathsetmacro{\thetaLower}{90 + \halfPowerAngle}
\pgfmathsetmacro{\beamWidth}{2.0 * \halfPowerAngle}
% Draw arc at the -3dB radius
\draw[rose, thick] (\thetaUpper:\RdBthree)
    arc (\thetaUpper:\thetaLower:\RdBthree);
% Radial lines showing beamwidth
\draw[rose, thin, dashed] (0,0) -- (\thetaUpper:\Rmax*0.85);
\draw[rose, thin, dashed] (0,0) -- (\thetaLower:\Rmax*0.85);
% Beamwidth label
\node[font=\tiny, rose, anchor=west]
    at ({(\RdBthree+0.15)*cos(90)}, {(\RdBthree+0.15)*sin(\thetaLower)})
    {BW $= \pgfmathprintnumber[fixed,precision=1]{\beamWidth}^\circ$};

% Dipole antenna
\draw[black!80, line width=2pt] (0, -0.35) -- (0, 0.35);
\filldraw[black] (0, 0.35) circle (1.5pt);
\filldraw[black] (0, -0.35) circle (1.5pt);

% Directivity readout: D = 2n+1 for sin^n pattern (approximate)
\pgfmathsetmacro{\Dval}{2*\nExp + 1}
\pgfmathsetmacro{\DdBi}{10 * ln(\Dval) / ln(10)}

% Info box
\node[anchor=south west, font=\footnotesize, fill=white, inner sep=4pt,
      draw=warmgrey!50, fill opacity=0.9, text opacity=1, align=left]
    at (-\Rmax-0.8, -\Rmax-0.3)
    {$f(\theta) = |\sin\theta|^{\pgfmathprintnumber[fixed,precision=1]{\nExp}}$\\[2pt]
     $D \approx \pgfmathprintnumber[fixed,precision=1]{\DdBi}$ dBi};

\end{tikzpicture}
\end{document}
