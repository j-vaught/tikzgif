% Example 10: RC Circuit Motor-Fan (Charge / Discharge)
% A DC motor with fan blades driven by an RC circuit.  The capacitor charges
% (fan speeds up) then the switch opens and the capacitor discharges through
% the motor (fan slows down), making the RC time constant physically intuitive.
% Parameter: \PARAM (normalized time t/tau, 0 to 10)
% Range: 0 to 10, recommended 80 frames at 20 fps
% Difficulty: intermediate
% Features demonstrated: circuit drawing, exponential curves, rotating fan,
%   two-phase animation, dual panels
\documentclass[tikz]{standalone}
\usepackage{tikz}
\usepackage{pgfplots}
\usepackage{xcolor}
\pgfplotsset{compat=1.18}

% USC Brand Colors
\definecolor{garnet}{HTML}{73000A}
\definecolor{rose}{HTML}{CC2E40}
\definecolor{atlantic}{HTML}{466A9F}
\definecolor{congaree}{HTML}{1F414D}
\definecolor{horseshoe}{HTML}{65780B}
\definecolor{grass}{HTML}{CED318}
\definecolor{honeycomb}{HTML}{A49137}
\definecolor{warmgrey}{HTML}{676156}
\definecolor{sandstorm}{HTML}{FFF2E3}

\begin{document}
\begin{tikzpicture}

% ====================================================================
% Bounding box — keeps every frame the same size
% ====================================================================
\useasboundingbox (-0.6,-1.4) rectangle (15.2,7.0);

% ====================================================================
% Electrical parameters
% ====================================================================
\pgfmathsetmacro{\Vs}{5}                         % source voltage
\pgfmathsetmacro{\ttau}{max(\PARAM, 0.001)}      % t/tau clamped

% --- Phase flag and clamped discharge time ---
\pgfmathsetmacro{\isCharging}{\ttau < 5 ? 1 : 0}
\pgfmathsetmacro{\dt}{max(\ttau - 5, 0)}          % clamped >=0 to avoid exp overflow

% --- Capacitor voltage ---
\pgfmathsetmacro{\VcCharge}{\Vs * (1 - exp(-\ttau))}
\pgfmathsetmacro{\VcAtFive}{\Vs * (1 - exp(-5))}
\pgfmathsetmacro{\VcDischarge}{\VcAtFive * exp(-\dt)}
\pgfmathsetmacro{\Vc}{\isCharging ? \VcCharge : \VcDischarge}

% --- Normalized current magnitude (for line thickness) ---
\pgfmathsetmacro{\IcCharge}{exp(-\ttau)}
\pgfmathsetmacro{\IcDischarge}{(\VcAtFive/\Vs) * exp(-\dt)}
\pgfmathsetmacro{\Ic}{\isCharging ? \IcCharge : \IcDischarge}

% --- Input voltage (step) ---
\pgfmathsetmacro{\Vin}{\isCharging ? \Vs : 0}

% ====================================================================
% Fan rotation angle  (integral of omega ~ Vc)
% ====================================================================
% rotScale = omega_max * tau.  Per-frame rotation at max speed must stay
% below 60 deg (half the 120 deg blade spacing) to avoid strobing.
% With 80 frames over [0,10]: du = 10/79 ≈ 0.127, so rotScale < 474.
% 270 gives ~34 deg/frame at peak, ~3 full rotations during charge.
\pgfmathsetmacro{\rotScale}{270}

% Charging:  angle = rotScale * [ t/tau + exp(-t/tau) - 1 ]
\pgfmathsetmacro{\angleCharge}{\rotScale * (\ttau + exp(-\ttau) - 1)}

% Value at the switching instant (t/tau = 5)
\pgfmathsetmacro{\angleAtFive}{\rotScale * (5 + exp(-5) - 1)}

% Discharging:  angle = angleAtFive + rotScale*(VcAtFive/Vs)*[1 - exp(-dt)]
\pgfmathsetmacro{\angleDischarge}{\angleAtFive + \rotScale * (\VcAtFive/\Vs) * (1 - exp(-\dt))}

\pgfmathsetmacro{\fanAngle}{\isCharging ? \angleCharge : \angleDischarge}

% ====================================================================
% LEFT PANEL — Circuit diagram  (shifted so origin is near center)
% ====================================================================
% Topology:  Vs --S--R--> node_A --+--(Motor+Fan)--+-- node_B --> Vs
%                                  |               |
%                                  +----( C )------+
% Motor and capacitor are in PARALLEL between node_A and node_B.
% ====================================================================
\begin{scope}[shift={(2.8,2.8)}]

  % ---------- Wires ----------
  \pgfmathsetmacro{\linewd}{0.6 + \Ic * 2.0}

  % Top rail: battery+ → switch → resistor → node_A
  \draw[black!80, line width=0.8pt] (-2.6, 2.0) -- (-2.6, 3.2) -- (-1.6, 3.2);
  % (switch gap drawn below)
  \draw[black!80, line width=0.8pt] (-0.4, 3.2) -- (0.6, 3.2);
  % (resistor drawn below)
  % node_A: junction where motor and capacitor branches split
  \draw[black!80, line width=0.8pt] (2.2, 3.2) -- (3.8, 3.2);
  % Motor branch: node_A down through motor to node_B
  \draw[black!80, line width=0.8pt] (2.6, 3.2) -- (2.6, 1.65);
  \draw[black!80, line width=0.8pt] (2.6, 0.15) -- (2.6, -1.2);
  % Capacitor branch: node_A down through cap to node_B
  \draw[black!80, line width=0.8pt] (3.8, 3.2) -- (3.8, -0.1);
  \draw[black!80, line width=0.8pt] (3.8, -0.7) -- (3.8, -1.2);
  % node_B: bottom junction, back to battery-
  \draw[black!80, line width=0.8pt] (2.6, -1.2) -- (3.8, -1.2);
  \draw[black!80, line width=0.8pt] (2.6, -1.2) -- (-2.6, -1.2) -- (-2.6, 0.8);
  % node_A dot
  \fill[black!80] (2.6, 3.2) circle (1.5pt);
  % Vertical tee to capacitor branch
  \fill[black!80] (3.8, 3.2) circle (1.5pt);
  % node_B dots
  \fill[black!80] (2.6, -1.2) circle (1.5pt);
  \fill[black!80] (3.8, -1.2) circle (1.5pt);

  % ---------- Current arrow (on top wire, thickness ~ |I|) ----------
  \pgfmathsetmacro{\arrwd}{max(0.3, \Ic * 2.5)}
  \draw[->, garnet, line width=\arrwd pt] (-0.2, 3.55) -- (0.6, 3.55);
  \node[above, font=\tiny, garnet] at (0.2, 3.6) {$i(t)$};

  % ---------- Battery ----------
  \draw[black!80, line width=1.2pt] (-2.95, 0.8) -- (-2.25, 0.8);  % short plate (-)
  \draw[black!80, line width=0.6pt] (-3.1, 1.2) -- (-2.1, 1.2);   % long plate (+)
  \draw[black!80, line width=0.8pt] (-2.6, 1.2) -- (-2.6, 2.0);   % wire up from +
  \draw[black!80, line width=0.8pt] (-2.6, 0.0) -- (-2.6, 0.8);   % wire down to -
  % polarity
  \node[font=\scriptsize] at (-3.25, 1.4) {$+$};
  \node[font=\scriptsize] at (-3.25, 0.6) {$-$};
  \node[left, font=\small] at (-3.15, 1.0) {$V_s$};

  % ---------- Switch ----------
  \fill[black!80] (-1.6, 3.2) circle (2.0pt);
  \fill[black!80] (-0.4, 3.2) circle (2.0pt);
  \ifnum\isCharging=1
    % Closed: straight bar
    \draw[garnet, line width=1.4pt] (-1.6, 3.2) -- (-0.4, 3.2);
    \node[above, font=\tiny, garnet] at (-1.0, 3.35) {closed};
  \else
    % Open: angled bar
    \draw[warmgrey, line width=1.4pt] (-1.6, 3.2) -- (-0.7, 3.75);
    \node[above, font=\tiny, warmgrey] at (-1.0, 3.75) {open};
  \fi
  \node[above, font=\small] at (-1.0, 3.85) {$S$};

  % ---------- Resistor (zigzag) ----------
  \draw[black!80, line width=0.8pt]
    (0.6,3.2) -- (0.8,3.2)
    -- (0.95,3.5) -- (1.15,2.9) -- (1.35,3.5) -- (1.55,2.9) -- (1.75,3.5) -- (1.95,2.9)
    -- (2.1,3.2) -- (2.2,3.2);
  \node[above, font=\small] at (1.4, 3.55) {$R$};

  % ---------- Motor (circle with M) — left branch ----------
  \draw[warmgrey, thick] (2.6, 0.9) circle (0.7);
  \node[font=\small, warmgrey] at (2.6, 0.9) {M};

  % ---------- Fan / Propeller (3 blades at 120° spacing) ----------
  \begin{scope}[shift={(2.6, 0.9)}]
    \foreach \k in {0,1,2} {
      \pgfmathsetmacro{\bladeAng}{\fanAngle + \k * 120}
      \fill[atlantic, opacity=0.85, rotate=\bladeAng]
        (0,0) -- (0.12, 0.25) -- (0, 1.05) -- (-0.12, 0.25) -- cycle;
    }
    % hub
    \fill[atlantic!70] (0,0) circle (0.1);
  \end{scope}

  % ---------- Capacitor — right branch ----------
  \draw[black!80, line width=1.2pt] (3.4, -0.1) -- (4.2, -0.1);  % top plate
  \draw[black!80, line width=1.2pt] (3.4, -0.7) -- (4.2, -0.7);  % bottom plate
  \node[right, font=\small] at (4.3, -0.4) {$C$};

  % charge-level fill
  \pgfmathsetmacro{\fillFrac}{\Vc / \Vs}
  \pgfmathsetmacro{\fillHt}{0.55 * \fillFrac}   % max gap ~ 0.55
  \fill[garnet!40, opacity=0.6]
    (3.45, -0.68) rectangle ({4.15}, {-0.68 + \fillHt});

  % ---------- Vc numeric readout ----------
  \node[below, font=\small] at (3.2, -1.5)
    {$V_C = \pgfmathprintnumber[fixed, precision=2]{\Vc}$\,V};

\end{scope}

% ====================================================================
% RIGHT PANEL — Voltage plot
% ====================================================================
\begin{axis}[
  at={(7.8cm, -0.6cm)},
  anchor=south west,
  width=7.0cm, height=6.8cm,
  domain=0:10,
  samples=200,
  axis lines=left,
  xlabel={$t/\tau$},
  ylabel={Voltage (V)},
  ymin=0, ymax=6.2,
  xmin=0, xmax=10.5,
  xtick={0,1,2,3,4,5,6,7,8,9,10},
  grid=major,
  grid style={warmgrey!20},
  every axis x label/.style={at={(ticklabel* cs:1.0)}, anchor=north west},
  every axis y label/.style={at={(ticklabel* cs:1.0)}, anchor=south east},
  clip=false,
]

  % ---- Vertical dashed line at switch event ----
  \draw[dashed, warmgrey!60, line width=0.6pt]
    (axis cs:5, 0) -- (axis cs:5, 6.2);

  % ---- Region labels ----
  \node[font=\scriptsize, congaree] at (axis cs:2.5, 5.9) {Charging};
  \node[font=\scriptsize, congaree] at (axis cs:7.5, 5.9) {Discharging};

  % ---- Vin step (full ghost) ----
  \addplot[atlantic, dashed, thick, forget plot]
    coordinates {(0,5)(4.99,5)(5,0)(10,0)};
  \node[font=\tiny, atlantic, anchor=south west] at (axis cs:0.1, 5.1) {$V_{in}$};

  % ---- Vc full ghost curve ----
  % charging portion
  \addplot[warmgrey!35, thick, domain=0:5, forget plot]
    {\Vs * (1 - exp(-x))};
  % discharging portion
  \addplot[warmgrey!35, thick, domain=5:10, forget plot]
    {\VcAtFive * exp(-(x - 5))};

  % ---- Vc active trace ----
  \pgfmathsetmacro{\clampedT}{min(\ttau, 10)}
  \pgfmathsetmacro{\traceEnd}{min(\clampedT, 5)}
  % charging part of the active trace
  \addplot[garnet, very thick, domain=0:\traceEnd, forget plot]
    {\Vs * (1 - exp(-x))};

  % discharging part of the active trace (only if past switch point)
  \pgfmathsetmacro{\showDischarge}{\ttau > 5 ? 1 : 0}
  \ifnum\showDischarge=1
    \addplot[garnet, very thick, domain=5:\clampedT, forget plot]
      {\VcAtFive * exp(-(x - 5))};
  \fi

  % ---- Current-time dot ----
  \addplot[only marks, mark=*, mark size=2.5pt, garnet, forget plot]
    coordinates {(\ttau, \Vc)};

  % Vc label near dot
  \node[font=\tiny, garnet, anchor=south west] at (axis cs:{\ttau + 0.15}, {\Vc + 0.15})
    {$V_C$};

  % ---- Asymptote ----
  \addplot[dashed, atlantic!40, thin, domain=0:5, forget plot] {\Vs};

\end{axis}

\end{tikzpicture}
\end{document}
