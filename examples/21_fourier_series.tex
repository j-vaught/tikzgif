% Example 21: Fourier Series Buildup
% Partial-sum convergence of a Fourier series to a square wave.
% Parameter: \PARAM (number of odd harmonics, range 1..41)
% Range: 1 to 41, recommended 21 frames
% Difficulty: intermediate
% Features demonstrated: Fourier series, convergence, spectrum bar chart,
%   error shading, Gibbs measurement, latest harmonic highlight, energy readout

\documentclass[border=5pt]{standalone}
\usepackage{tikz}
\usepackage{pgfplots}
\usepackage{amsmath}
\pgfplotsset{compat=1.18}
\usepgfplotslibrary{fillbetween}

% USC Brand Colors
\definecolor{garnet}{HTML}{73000A}
\definecolor{rose}{HTML}{CC2E40}
\definecolor{atlantic}{HTML}{466A9F}
\definecolor{congaree}{HTML}{1F414D}
\definecolor{horseshoe}{HTML}{65780B}
\definecolor{grass}{HTML}{CED318}
\definecolor{honeycomb}{HTML}{A49137}
\definecolor{warmgrey}{HTML}{676156}
\definecolor{sandstorm}{HTML}{FFF2E3}

\begin{document}
\begin{tikzpicture}

\pgfmathtruncatemacro{\Nterms}{\PARAM}

% Compute energy fraction: sum of 1/n^2 for odd n up to N, divided by pi^2/8
\pgfmathsetmacro{\energySum}{%
    (\Nterms >= 1  ? 1.0   : 0) +
    (\Nterms >= 3  ? 1.0/9   : 0) +
    (\Nterms >= 5  ? 1.0/25  : 0) +
    (\Nterms >= 7  ? 1.0/49  : 0) +
    (\Nterms >= 9  ? 1.0/81  : 0) +
    (\Nterms >= 11 ? 1.0/121 : 0) +
    (\Nterms >= 13 ? 1.0/169 : 0) +
    (\Nterms >= 15 ? 1.0/225 : 0) +
    (\Nterms >= 17 ? 1.0/289 : 0) +
    (\Nterms >= 19 ? 1.0/361 : 0) +
    (\Nterms >= 21 ? 1.0/441 : 0)}
\pgfmathsetmacro{\energyFrac}{min(\energySum / 1.2337, 1.0)}
\pgfmathsetmacro{\energyPct}{\energyFrac * 100}

%% ---- MAIN WAVEFORM PLOT ----
\begin{axis}[
    name=waveax,
    width=10cm, height=6cm,
    xmin=-3.5, xmax=10.5, ymin=-1.6, ymax=1.6,
    xlabel={$x$}, ylabel={$f(x)$},
    title={Fourier Partial Sum: $N = \Nterms$ odd harmonics},
    grid=major, grid style={warmgrey!20}, thick,
    every axis title/.style={font=\normalsize},
    tick label style={font=\small},
    xtick={-3.14159, 0, 3.14159, 6.28318, 9.42478},
    xticklabels={$-\pi$, $0$, $\pi$, $2\pi$, $3\pi$},
]
    % Target square wave
    \addplot[domain=-3.5:10.5, samples=500, name path=sqwave,
        dashed, black!40, thin]
        {(sin(deg(x)) >= 0) ? 1 : -1};

    % Fourier partial sum
    \addplot[domain=-3.5:10.5, samples=400, name path=fourier,
        atlantic, very thick]
        {(4/3.14159) * (
            (\Nterms >= 1  ? sin(deg(1*x))/1   : 0) +
            (\Nterms >= 3  ? sin(deg(3*x))/3   : 0) +
            (\Nterms >= 5  ? sin(deg(5*x))/5   : 0) +
            (\Nterms >= 7  ? sin(deg(7*x))/7   : 0) +
            (\Nterms >= 9  ? sin(deg(9*x))/9   : 0) +
            (\Nterms >= 11 ? sin(deg(11*x))/11 : 0) +
            (\Nterms >= 13 ? sin(deg(13*x))/13 : 0) +
            (\Nterms >= 15 ? sin(deg(15*x))/15 : 0) +
            (\Nterms >= 17 ? sin(deg(17*x))/17 : 0) +
            (\Nterms >= 19 ? sin(deg(19*x))/19 : 0) +
            (\Nterms >= 21 ? sin(deg(21*x))/21 : 0) +
            (\Nterms >= 23 ? sin(deg(23*x))/23 : 0) +
            (\Nterms >= 25 ? sin(deg(25*x))/25 : 0) +
            (\Nterms >= 27 ? sin(deg(27*x))/27 : 0) +
            (\Nterms >= 29 ? sin(deg(29*x))/29 : 0) +
            (\Nterms >= 31 ? sin(deg(31*x))/31 : 0) +
            (\Nterms >= 33 ? sin(deg(33*x))/33 : 0) +
            (\Nterms >= 35 ? sin(deg(35*x))/35 : 0) +
            (\Nterms >= 37 ? sin(deg(37*x))/37 : 0) +
            (\Nterms >= 39 ? sin(deg(39*x))/39 : 0) +
            (\Nterms >= 41 ? sin(deg(41*x))/41 : 0)
        )};

    % Error shading between partial sum and square wave
    \addplot[fill=rose, fill opacity=0.12] fill between[of=fourier and sqwave];

    % Latest harmonic highlight (thin garnet curve showing last added term)
    \pgfmathparse{\Nterms >= 3 ? 1 : 0}
    \ifnum\pgfmathresult=1
        \addplot[domain=-3.5:10.5, samples=300, garnet, thin]
            {(4/3.14159) * sin(deg(\Nterms*x))/\Nterms};
    \fi

    % Gibbs overshoot measurement
    \pgfmathparse{\Nterms >= 3 ? 1 : 0}
    \ifnum\pgfmathresult=1
        \draw[<-, garnet, thin]
            (axis cs:0.15, 1.18) -- ++(axis direction cs:0.8, 0.15)
            node[right, font=\tiny, garnet] {Gibbs $\approx 9\%$};
    \fi
\end{axis}

%% ---- SPECTRUM BAR CHART ----
\begin{axis}[
    at={(waveax.east)}, anchor=west, xshift=8mm,
    width=4cm, height=6cm,
    ybar, bar width=5pt,
    ymin=0, ymax=1.4,
    xmin=0, xmax=22,
    xlabel={\tiny harmonic $n$},
    ylabel={\tiny $|b_n|$},
    title={\tiny Spectrum},
    every axis title/.style={font=\tiny},
    tick label style={font=\tiny},
    label style={font=\tiny},
    xtick={1, 5, 9, 13, 17, 21},
    ymajorgrids=true, grid style={warmgrey!15},
]
    % Bars: active harmonics in atlantic, inactive in warmgrey
    \foreach \hn in {1, 3, 5, 7, 9, 11, 13, 15, 17, 19, 21} {
        \pgfmathsetmacro{\barH}{4.0 / (3.14159 * \hn)}
        \pgfmathparse{\hn <= \Nterms ? 1 : 0}
        \ifnum\pgfmathresult=1
            \addplot[fill=atlantic!70, draw=atlantic] coordinates {(\hn, \barH)};
        \else
            \addplot[fill=warmgrey!15, draw=warmgrey!30] coordinates {(\hn, \barH)};
        \fi
    }
\end{axis}

% Info box
\node[anchor=north west, fill=white, fill opacity=0.9, text opacity=1,
      draw=warmgrey!50, inner sep=4pt, font=\footnotesize, align=left]
    at ([shift={(0.2cm,-0.2cm)}]waveax.north west)
    {$N = \Nterms$ terms\\[1pt]
     Energy: $\pgfmathprintnumber[fixed,precision=1]{\energyPct}\%$};

\end{tikzpicture}
\end{document}
