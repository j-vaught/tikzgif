% Example 28: 3D Object Rotation
% Rotates a wireframe cube using tikz-3dplot library.
% Parameter: \PARAM (azimuthal viewing angle in degrees, range 0.0..360.0)
% Range: 0.0 to 360.0, recommended 72 frames
% Difficulty: intermediate
% Features demonstrated: 3D rotation, cube wireframe, viewing angle variation

\documentclass[border=10pt,tikz]{standalone}
\usepackage{tikz}
\usepackage{tikz-3dplot}
\usepackage{amsmath}
\usetikzlibrary{arrows.meta, calc}

% USC Brand Colors
\definecolor{garnet}{HTML}{73000A}
\definecolor{rose}{HTML}{CC2E40}
\definecolor{atlantic}{HTML}{466A9F}
\definecolor{congaree}{HTML}{1F414D}
\definecolor{horseshoe}{HTML}{65780B}
\definecolor{grass}{HTML}{CED318}
\definecolor{honeycomb}{HTML}{A49137}
\definecolor{warmgrey}{HTML}{676156}
\definecolor{sandstorm}{HTML}{FFF2E3}

\begin{document}

\pgfmathsetmacro{\viewPhi}{\PARAM}
\pgfmathsetmacro{\viewTheta}{70.0}

\tdplotsetmaincoords{\viewTheta}{\viewPhi}

\begin{tikzpicture}[tdplot_main_coords, >=Stealth, scale=1.5]

\pgfmathsetmacro{\s}{1.5}

% Axes
\draw[->, thick, garnet]   (0,0,0) -- (2.5,0,0) node[anchor=north east]{$x$};
\draw[->, thick, horseshoe] (0,0,0) -- (0,2.5,0) node[anchor=north west]{$y$};
\draw[->, thick, atlantic]  (0,0,0) -- (0,0,2.5) node[anchor=south]{$z$};

% Vertices
\coordinate (A) at (-\s, -\s, -\s);
\coordinate (B) at ( \s, -\s, -\s);
\coordinate (C) at ( \s,  \s, -\s);
\coordinate (D) at (-\s,  \s, -\s);
\coordinate (E) at (-\s, -\s,  \s);
\coordinate (F) at ( \s, -\s,  \s);
\coordinate (G) at ( \s,  \s,  \s);
\coordinate (H) at (-\s,  \s,  \s);

% Edges
\draw[thick, black!70] (A) -- (B) -- (C) -- (D) -- cycle;
\draw[thick, black!70] (E) -- (F) -- (G) -- (H) -- cycle;
\draw[thick, black!70] (A) -- (E);
\draw[thick, black!70] (B) -- (F);
\draw[thick, black!70] (C) -- (G);
\draw[thick, black!70] (D) -- (H);

% Face shading
\fill[atlantic!15, opacity=0.4] (E) -- (F) -- (G) -- (H) -- cycle;
\fill[garnet!10, opacity=0.3] (A) -- (B) -- (F) -- (E) -- cycle;
\fill[horseshoe!10, opacity=0.3] (B) -- (C) -- (G) -- (F) -- cycle;

% Body diagonal
\draw[dashed, rose, thin] (A) -- (G);

% Center
\filldraw[garnet] (0, 0, 0) circle (2pt);

% Label
\node[anchor=north, font=\footnotesize, fill=white, inner sep=2pt, draw=warmgrey]
    at (0, 0, -\s-0.6)
    {$\phi = \pgfmathprintnumber[fixed, precision=0]{\viewPhi}^\circ$};

\end{tikzpicture}
\end{document}
