% Example 27: Root Locus Animation
% Animates poles of 1 + K*G(s) = 0 where G(s) = 1/((s+1)(s+3)).
% Parameter: \PARAM (loop gain, range 0.0..20.0)
% Range: 0.0 to 20.0, recommended 40 frames
% Difficulty: advanced
% Features demonstrated: control systems, root locus, pole movement,
%   damping lines, frequency circles, breakaway point, pole ghost trail

\documentclass[border=8pt,tikz]{standalone}
\usepackage{tikz}
\usepackage{amsmath}
\usetikzlibrary{arrows.meta, calc}

% USC Brand Colors
\definecolor{garnet}{HTML}{73000A}
\definecolor{rose}{HTML}{CC2E40}
\definecolor{atlantic}{HTML}{466A9F}
\definecolor{congaree}{HTML}{1F414D}
\definecolor{horseshoe}{HTML}{65780B}
\definecolor{grass}{HTML}{CED318}
\definecolor{honeycomb}{HTML}{A49137}
\definecolor{warmgrey}{HTML}{676156}
\definecolor{sandstorm}{HTML}{FFF2E3}

\begin{document}
\begin{tikzpicture}[>=Stealth, every node/.style={font=\small}]

\pgfmathsetmacro{\Kval}{\PARAM}

% Axes
\draw[->, thick] (-5.5, 0) -- (1.5, 0) node[right] {$\mathrm{Re}$};
\draw[->, thick] (0, -4.5) -- (0, 4.5) node[above] {$\mathrm{Im}$};

% Grid
\foreach \x in {-5,-4,-3,-2,-1,1} {
    \draw[warmgrey!20] (\x, -4.2) -- (\x, 4.2);
    \node[below, font=\tiny, warmgrey] at (\x, -4.3) {$\x$};
}
\foreach \y in {-4,-3,-2,-1,1,2,3,4} {
    \draw[warmgrey!20] (-5.3, \y) -- (1.3, \y);
}

% Constant damping ratio lines (zeta = 0.3, 0.5, 0.7)
\foreach \zetaL/\zetaLab in {0.3/0.3, 0.5/0.5, 0.7/0.7} {
    \pgfmathsetmacro{\thetaL}{acos(\zetaL)}
    \pgfmathsetmacro{\endR}{5.0}
    \pgfmathsetmacro{\endX}{-\endR * cos(\thetaL)}
    \pgfmathsetmacro{\endYp}{\endR * sin(\thetaL)}
    \pgfmathsetmacro{\endYn}{-\endR * sin(\thetaL)}
    \draw[dashed, horseshoe!50, very thin] (0,0) -- (\endX, \endYp);
    \draw[dashed, horseshoe!50, very thin] (0,0) -- (\endX, \endYn);
    \node[font=\tiny, horseshoe!60, anchor=south west] at (\endX, \endYp)
        {$\zeta\!=\!\zetaLab$};
}

% Natural frequency semicircles (wn = 1, 2, 3, 4)
\foreach \wnR in {1, 2, 3, 4} {
    \draw[dashed, warmgrey!25, very thin] (0,0) ++(180:\wnR) arc (180:90:\wnR);
    \draw[dashed, warmgrey!25, very thin] (0,0) ++(180:\wnR) arc (180:270:\wnR);
    \node[font=\tiny, warmgrey!40, anchor=south] at (0, \wnR) {$\wnR$};
}

% Open-loop poles (crosses)
\draw[thick] (-1, -0.15) -- (-1, 0.15);
\draw[thick] (-1.15, 0) -- (-0.85, 0);
\node[below=2mm, font=\footnotesize] at (-1, 0) {$-1$};
\draw[thick] (-3, -0.15) -- (-3, 0.15);
\draw[thick] (-3.15, 0) -- (-2.85, 0);
\node[below=2mm, font=\footnotesize] at (-3, 0) {$-3$};

% Breakaway point at s = -2 (K = 1)
\node[mark size=3pt, font=\tiny] at (-2, 0) {};
\filldraw[congaree] (-2, -0.08) -- (-1.92, 0.08) -- (-2.08, 0.08) -- cycle;
\filldraw[congaree] (-2, 0.08) -- (-1.92, -0.08) -- (-2.08, -0.08) -- cycle;
\node[font=\tiny, congaree, anchor=north west] at (-1.9, -0.15)
    {Breakaway};

% Characteristic equation: s^2 + 4s + 3 + K = 0
% s = -2 +/- sqrt(1 - K)

% Pole ghost trail: show faded circles at previous K values
\foreach \gK/\gop in {0.2/8, 0.5/12, 1.0/16, 2.0/20, 4.0/24, 8.0/28, 12.0/32} {
    \pgfmathparse{\gK <= \Kval ? 1 : 0}
    \ifnum\pgfmathresult=1
        \pgfmathparse{\gK <= 1.0 ? 1 : 0}
        \ifnum\pgfmathresult=1
            \pgfmathsetmacro{\gspread}{sqrt(max(1 - \gK, 0))}
            \fill[warmgrey, opacity=0.\gop]
                ({-2 + \gspread}, 0) circle (2pt);
            \fill[warmgrey, opacity=0.\gop]
                ({-2 - \gspread}, 0) circle (2pt);
        \else
            \pgfmathsetmacro{\gimag}{sqrt(\gK - 1)}
            \fill[warmgrey, opacity=0.\gop]
                (-2, \gimag) circle (2pt);
            \fill[warmgrey, opacity=0.\gop]
                (-2, -\gimag) circle (2pt);
        \fi
    \fi
}

% Real-axis segment (K=0 to K=1)
\pgfmathparse{\Kval > 0.01 ? 1 : 0}
\ifnum\pgfmathresult=1
    \pgfmathparse{min(\Kval, 1.0)}
    \pgfmathsetmacro{\Kclamp}{\pgfmathresult}
    \pgfmathsetmacro{\realSpread}{sqrt(max(1 - \Kclamp, 0))}
    \draw[atlantic, thick] (-1, 0) -- ({-2 + \realSpread}, 0);
    \draw[atlantic, thick] (-3, 0) -- ({-2 - \realSpread}, 0);
\fi

% Imaginary branches (K > 1)
\pgfmathparse{\Kval > 1.01 ? 1 : 0}
\ifnum\pgfmathresult=1
    \pgfmathsetmacro{\imMax}{sqrt(\Kval - 1)}
    \draw[garnet, thick, samples=60, domain=0:\imMax, variable=\y]
        plot ({-2}, {\y});
    \draw[garnet, thick, samples=60, domain=0:\imMax, variable=\y]
        plot ({-2}, {-\y});
\fi

% Current pole positions
\pgfmathparse{\Kval <= 1.0 ? 1 : 0}
\ifnum\pgfmathresult=1
    \pgfmathsetmacro{\poleA}{-2 + sqrt(max(1 - \Kval, 0))}
    \pgfmathsetmacro{\poleB}{-2 - sqrt(max(1 - \Kval, 0))}
    \filldraw[garnet] (\poleA, 0) circle (3.5pt);
    \filldraw[garnet] (\poleB, 0) circle (3.5pt);
\fi
\pgfmathparse{\Kval > 1.0 ? 1 : 0}
\ifnum\pgfmathresult=1
    \pgfmathsetmacro{\imPart}{sqrt(\Kval - 1)}
    \filldraw[garnet] (-2, \imPart) circle (3.5pt);
    \filldraw[garnet] (-2, -\imPart) circle (3.5pt);

    % Damping line from origin to upper pole
    \draw[dashed, rose, thin] (0, 0) -- (-2, \imPart);

    % Compute current zeta and wn
    \pgfmathsetmacro{\curWn}{sqrt(4 + \imPart*\imPart)}
    \pgfmathsetmacro{\curZeta}{2.0 / \curWn}
    \node[font=\tiny, rose, anchor=south west]
        at (-1.8, {\imPart*0.5})
        {$\zeta\!=\!\pgfmathprintnumber[fixed,precision=2]{\curZeta}$};
\fi

% K label
\node[anchor=north west, font=\normalsize, fill=white, inner sep=3pt,
      draw=warmgrey!50]
    at (-5.3, 4.3)
    {$K = \pgfmathprintnumber[fixed, precision=1]{\Kval}$};

% Transfer function label
\node[anchor=south west, font=\footnotesize, text=warmgrey!70!black]
    at (-5.3, -4.3)
    {$G(s) = \dfrac{1}{(s+1)(s+3)}$};

% Characteristic equation
\node[anchor=south east, font=\tiny, text=warmgrey!60!black,
      fill=white, fill opacity=0.8, text opacity=1, inner sep=2pt]
    at (1.3, -4.3)
    {$s^2 + 4s + 3 + K = 0$};

\end{tikzpicture}
\end{document}
