% Example 30: Two-Source Wave Interference
% Two coherent sources create an interference pattern with phase difference.
% Parameter: \PARAM (phase difference in radians, range 0.0..6.283185)
% Range: 0.0 to 6.283185, recommended 30 frames
% Difficulty: advanced
% Features demonstrated: wave interference, superposition, phase effects

\documentclass[border=5pt,tikz]{standalone}
\usepackage{tikz}
\usepackage{amsmath}
\usetikzlibrary{calc}

% USC Brand Colors
\definecolor{garnet}{HTML}{73000A}
\definecolor{rose}{HTML}{CC2E40}
\definecolor{atlantic}{HTML}{466A9F}
\definecolor{congaree}{HTML}{1F414D}
\definecolor{horseshoe}{HTML}{65780B}
\definecolor{grass}{HTML}{CED318}
\definecolor{honeycomb}{HTML}{A49137}
\definecolor{warmgrey}{HTML}{676156}
\definecolor{sandstorm}{HTML}{FFF2E3}

\begin{document}
\begin{tikzpicture}

\pgfmathsetmacro{\deltaPhi}{\PARAM}
\pgfmathsetmacro{\separation}{2.0}
\pgfmathsetmacro{\halfSep}{\separation / 2}
\pgfmathsetmacro{\waveK}{3.0}

\pgfmathsetmacro{\xA}{-\halfSep}
\pgfmathsetmacro{\xB}{\halfSep}

% Interference pattern (pixelated color map)
\foreach \xi in {-4.0, -3.8, ..., 3.8} {
    \foreach \yi in {-3.0, -2.8, ..., 2.8} {
        \pgfmathsetmacro{\px}{\xi + 0.1}
        \pgfmathsetmacro{\py}{\yi + 0.1}
        \pgfmathsetmacro{\rA}{sqrt((\px - \xA)*(\px - \xA) + \py*\py)}
        \pgfmathsetmacro{\rB}{sqrt((\px - \xB)*(\px - \xB) + \py*\py)}
        \pgfmathsetmacro{\valA}{cos(\waveK * \rA * 180 / 3.14159)}
        \pgfmathsetmacro{\valB}{cos(\waveK * \rB * 180 / 3.14159 + \deltaPhi * 180 / 3.14159)}
        \pgfmathsetmacro{\amp}{\valA + \valB}
        \pgfmathsetmacro{\intensity}{(\amp * \amp) / 4.0}
        \pgfmathsetmacro{\colorVal}{100 * \intensity}
        \fill[atlantic!\colorVal!black] (\xi, \yi) rectangle +(0.2, 0.2);
    }
}

% Source markers
\filldraw[white] (\xA, 0) circle (3pt);
\filldraw[white] (\xB, 0) circle (3pt);
\node[below=3mm, font=\tiny, white] at (\xA, 0) {$S_1$};
\node[below=3mm, font=\tiny, white] at (\xB, 0) {$S_2$};

% Faint wavefronts
\foreach \r in {0.5, 1.0, 1.5, 2.0, 2.5, 3.0, 3.5, 4.0} {
    \draw[white, opacity=0.15, thin] (\xA, 0) circle (\r);
    \draw[white, opacity=0.15, thin] (\xB, 0) circle (\r);
}

% Title
\node[anchor=north west, font=\footnotesize, fill=black!70, text=white, inner sep=3pt]
    at (-4, 3)
    {$\Delta\phi = \pgfmathprintnumber[fixed, precision=2]{\deltaPhi}$ rad};

\end{tikzpicture}
\end{document}
