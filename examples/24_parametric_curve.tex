% Example 24: Parametric Curve Tracing (Lissajous)
% A point traces a Lissajous figure as the parameter sweeps from 0 to 2π.
% Parameter: \PARAM (maximum t value, range 0.1..6.283185 rad)
% Range: 0.1 to 6.283185, recommended 48 frames
% Difficulty: intermediate
% Features demonstrated: parametric curves, Lissajous figures, curve tracing
\documentclass[border=10pt,tikz]{standalone}
\usepackage{tikz}
\usepackage{amsmath}
\usetikzlibrary{arrows.meta}

% USC Brand Colors
\definecolor{garnet}{HTML}{73000A}
\definecolor{rose}{HTML}{CC2E40}
\definecolor{atlantic}{HTML}{466A9F}
\definecolor{congaree}{HTML}{1F414D}
\definecolor{horseshoe}{HTML}{65780B}
\definecolor{grass}{HTML}{CED318}
\definecolor{honeycomb}{HTML}{A49137}
\definecolor{warmgrey}{HTML}{676156}
\definecolor{sandstorm}{HTML}{FFF2E3}

\begin{document}
\begin{tikzpicture}[>=Stealth]

\pgfmathsetmacro{\tmax}{\PARAM}
\pgfmathsetmacro{\fa}{3.0}
\pgfmathsetmacro{\fb}{2.0}
\pgfmathsetmacro{\phd}{1.5708}
\pgfmathsetmacro{\amp}{3.0}

% Axes
\draw[->, warmgrey!60, thick] (-\amp-0.5, 0) -- (\amp+0.5, 0) node[right, font=\small]{$x$};
\draw[->, warmgrey!60, thick] (0, -\amp-0.5) -- (0, \amp+0.5) node[above, font=\small]{$y$};

% Ghost of full curve
\draw[warmgrey!20, thin, samples=300, domain=0:360, variable=\t]
    plot ({\amp * sin(\fa*\t + \phd*180/3.14159)}, {\amp * sin(\fb*\t)});

% Traced curve up to tmax
\pgfmathsetmacro{\tmaxDeg}{\tmax * 180 / 3.14159}
\pgfmathsetmacro{\trailStart}{max(\tmaxDeg - 60, 0)}

\draw[atlantic!30, thick, samples=100, domain=0:\trailStart, variable=\t]
    plot ({\amp * sin(\fa*\t + \phd*180/3.14159)}, {\amp * sin(\fb*\t)});
\draw[atlantic!70!black, very thick, samples=80, domain=\trailStart:\tmaxDeg, variable=\t]
    plot ({\amp * sin(\fa*\t + \phd*180/3.14159)}, {\amp * sin(\fb*\t)});

% Current point
\pgfmathsetmacro{\currX}{\amp * sin(\fa*\tmaxDeg + \phd*180/3.14159)}
\pgfmathsetmacro{\currY}{\amp * sin(\fb*\tmaxDeg)}
\filldraw[garnet!80!black] (\currX, \currY) circle (4pt);
\draw[dashed, garnet!40, thin] (\currX, \currY) -- (\currX, 0);
\draw[dashed, garnet!40, thin] (\currX, \currY) -- (0, \currY);

% Label
\node[anchor=north west, font=\small, fill=white, inner sep=3pt, draw=warmgrey!40]
    at (-\amp-0.3, \amp+0.4)
    {$\begin{aligned}
        x &= \sin(\pgfmathprintnumber[fixed,precision=0]{\fa}\,t
              + \pgfmathprintnumber[fixed,precision=2]{\phd}) \\
        y &= \sin(\pgfmathprintnumber[fixed,precision=0]{\fb}\,t)
     \end{aligned}$};

\node[anchor=south east, font=\footnotesize, fill=white, inner sep=2pt]
    at (\amp+0.3, -\amp-0.3)
    {$t = \pgfmathprintnumber[fixed, precision=2]{\tmax}$};

\end{tikzpicture}
\end{document}
