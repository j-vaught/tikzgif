% Example 02: Bouncing Ball
% Parabolic trajectory with shadow on the ground plane.
% Parameter: \PARAM (time in range 0..1, one full bounce cycle)
% Range: 0 to 1, recommended 40 frames
% Difficulty: beginner
% Features demonstrated: mathematical expressions in coordinates, shadows
\documentclass[tikz]{standalone}
\usepackage{tikz}
\begin{document}
\begin{tikzpicture}
  \useasboundingbox (-0.5,-0.5) rectangle (8.5,5.5);
  % Ground
  \fill[black!10] (-0.5,-0.5) rectangle (8.5,0);
  \draw[thick] (-0.5,0) -- (8.5,0);
  % Horizontal position: linear in time
  \pgfmathsetmacro{\xpos}{\PARAM * 8}
  % Vertical position: parabolic (peak at t=0.5)
  \pgfmathsetmacro{\ypos}{4 * 4 * \PARAM * (1 - \PARAM)}
  % Shadow size inversely proportional to height
  \pgfmathsetmacro{\shadowscale}{1 - 0.6 * 4 * \PARAM * (1 - \PARAM)}
  \pgfmathsetmacro{\shadowalpha}{80 - 50 * 4 * \PARAM * (1 - \PARAM) / 4}
  % Shadow on the ground
  \fill[black!\shadowalpha, opacity=0.4]
    (\xpos, 0) ellipse ({\shadowscale * 0.5} and 0.08);
  % Ball
  \shade[ball color=red!70!black] (\xpos, \ypos) circle (0.35);
  % Trajectory guide (faint)
  \draw[dashed, gray!50, thin]
    plot[domain=0:1, samples=50, variable=\t]
    ({\t * 8}, {4 * 4 * \t * (1 - \t)});
  % Label
  \node[below right] at (0.2,5.2) {\small $t = \PARAM$};
\end{tikzpicture}
\end{document}
