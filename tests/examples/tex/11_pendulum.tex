% Example 11: Pendulum Motion
% Simple pendulum swinging with a fading trail of past positions.
% Parameter: \PARAM (time in seconds, 0 to 6.28 for one full period approx)
% Range: 0 to 6.28, recommended 60 frames
% Difficulty: intermediate
% Features demonstrated: trigonometric motion, trail/ghost effect, physics
\documentclass[tikz]{standalone}
\usepackage{tikz}
\begin{document}
\begin{tikzpicture}
  \useasboundingbox (-5,-6) rectangle (5,1.5);

  \pgfmathsetmacro{\tval}{\PARAM}
  \pgfmathsetmacro{\length}{4.5}     % pendulum length
  \pgfmathsetmacro{\thetamax}{40}    % max angle in degrees
  \pgfmathsetmacro{\omega}{2.214}    % angular frequency (sqrt(g/L) for L~2m)

  % Current angle: theta(t) = theta_max * cos(omega * t)
  \pgfmathsetmacro{\theta}{\thetamax * cos(deg(\omega * \tval))}

  % Pivot point
  \coordinate (pivot) at (0, 0.5);
  \fill[black] (pivot) circle (3pt);
  \draw[thick] (-1.5, 0.5) -- (1.5, 0.5);
  \foreach \xx in {-1.4,-1.1,...,1.5} {
    \draw[thin, gray] (\xx, 0.5) -- ({\xx + 0.3}, 0.8);
  }

  % Trail: draw past positions as fading circles
  \foreach \k in {1,2,...,12} {
    \pgfmathsetmacro{\tpast}{\tval - \k * 0.05}
    \pgfmathsetmacro{\thpast}{\thetamax * cos(deg(\omega * \tpast))}
    \pgfmathsetmacro{\xpast}{\length * sin(\thpast)}
    \pgfmathsetmacro{\ypast}{0.5 - \length * cos(\thpast)}
    \pgfmathsetmacro{\opac}{0.4 * (1 - \k/13)}
    \fill[red!60!black, opacity=\opac] (\xpast, \ypast) circle (4pt);
  }

  % Bob position
  \pgfmathsetmacro{\xbob}{\length * sin(\theta)}
  \pgfmathsetmacro{\ybob}{0.5 - \length * cos(\theta)}

  % Rod
  \draw[thick, black!70] (pivot) -- (\xbob, \ybob);

  % Bob
  \shade[ball color=red!70!black] (\xbob, \ybob) circle (8pt);

  % Angle arc
  \draw[thin, gray, ->] (0, {0.5 - 1.2}) arc (-90:{-90 + \theta}:1.2);
  \pgfmathsetmacro{\labelangle}{-90 + \theta/2}
  \node[font=\tiny, gray] at ({0.8 * sin(\theta/2)}, {0.5 - 0.8 * cos(\theta/2)})
    {$\theta$};

  % Labels
  \node[below, font=\small] at (0, -5.5)
    {$\theta = \pgfmathprintnumber[fixed, precision=1]{\theta}^\circ$
     \quad $t = \pgfmathprintnumber[fixed, precision=2]{\tval}$ s};
\end{tikzpicture}
\end{document}
