% Example 06: Sorting Algorithm Visualization
% Bars showing values being progressively sorted.
% Instead of running bubble sort in TeX (which is fragile), we use
% the parameter directly as a permutation index into a pre-shuffled
% sequence that smoothly transitions from shuffled to sorted.
% Parameter: \PARAM (interpolation factor, 0=shuffled to 1=sorted)
% Range: 0 to 1, recommended 30 frames
% Difficulty: intermediate
% Features demonstrated: computed coordinates, color mapping, bar charts
\documentclass[tikz]{standalone}
\usepackage{tikz}
\begin{document}
\begin{tikzpicture}
  \useasboundingbox (-0.5,-1) rectangle (9.5,7);

  \pgfmathsetmacro{\progress}{\PARAM}

  % Unsorted order:  6, 2, 8, 4, 1, 7, 3, 5  (positions 0..7)
  % Sorted order:    1, 2, 3, 4, 5, 6, 7, 8  (positions 0..7)
  % We interpolate bar positions from unsorted slot to sorted slot.

  % Value at each unsorted position
  \foreach \val/\unsortedPos/\sortedPos in {
    6/0/5, 2/1/1, 8/2/7, 4/3/3, 1/4/0, 7/5/6, 3/6/2, 5/7/4%
  } {
    % Interpolate horizontal position
    \pgfmathsetmacro{\xpos}{(\unsortedPos * (1 - \progress) + \sortedPos * \progress) * 1.15 + 0.5}
    \pgfmathsetmacro{\barht}{\val * 0.75}

    % Color based on how far from final position
    \pgfmathsetmacro{\dist}{abs(\unsortedPos - \sortedPos)}
    \pgfmathtruncatemacro{\colorval}{min(100, max(20, 100 - \dist * 12))}

    \fill[blue!\colorval!red] (\xpos - 0.4, 0) rectangle (\xpos + 0.4, \barht);
    \draw[black!80, thick] (\xpos - 0.4, 0) rectangle (\xpos + 0.4, \barht);
    \node[above, font=\small\bfseries] at (\xpos, \barht) {\val};
  }

  % Ground line
  \draw[thick] (-0.3, 0) -- (9.3, 0);

  % Progress indicator
  \pgfmathtruncatemacro{\pctdone}{\progress * 100}
  \node[font=\small] at (4.5, -0.6) {Progress: \pctdone\%};
  \node[font=\large\bfseries] at (4.5, 6.5) {Sorting Animation};
\end{tikzpicture}
\end{document}
