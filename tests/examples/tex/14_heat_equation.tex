% Example 14: Heat Equation (1D Diffusion)
% A color-coded bar showing temperature distribution evolving over time.
% Initial condition: hot spot in the center, cold at edges.
% We approximate the solution: T(x,t) = sum of Fourier modes decaying.
% Parameter: \PARAM (normalized time alpha*t/L^2, 0 to 0.5)
% Range: 0 to 0.5, recommended 50 frames
% Difficulty: advanced
% Features demonstrated: Fourier series computation, color mapping, scientific viz
\documentclass[tikz]{standalone}
\usepackage{tikz}
\usepackage{pgfplots}
\usepackage{xcolor}
\pgfplotsset{compat=1.18}

% USC Brand Colors
\definecolor{garnet}{HTML}{73000A}
\definecolor{rose}{HTML}{CC2E40}
\definecolor{atlantic}{HTML}{466A9F}
\definecolor{congaree}{HTML}{1F414D}
\definecolor{horseshoe}{HTML}{65780B}
\definecolor{grass}{HTML}{CED318}
\definecolor{honeycomb}{HTML}{A49137}
\definecolor{warmgrey}{HTML}{676156}
\definecolor{sandstorm}{HTML}{FFF2E3}

\begin{document}
\begin{tikzpicture}
  \pgfmathsetmacro{\tval}{\PARAM}  % normalized diffusion time

  % Spatial resolution
  \pgfmathtruncatemacro{\nx}{60}
  \pgfmathsetmacro{\barwidth}{10}
  \pgfmathsetmacro{\barheight}{1.2}
  \pgfmathsetmacro{\dx}{\barwidth / \nx}

  % Number of Fourier terms to sum
  \pgfmathtruncatemacro{\nterms}{10}

  % Initial condition: T(x,0) = sin(pi*x/L) (first mode dominant)
  % Exact solution for sin initial condition with homogeneous Dirichlet BCs:
  % T(x,t) = sin(pi*x) * exp(-pi^2 * t) + 0.5*sin(3*pi*x)*exp(-9*pi^2*t) + ...

  % Draw the color bar
  \foreach \i in {0,...,\numexpr\nx-1} {
    \pgfmathsetmacro{\xfrac}{(\i + 0.5) / \nx}  % x/L in [0,1]

    % Compute temperature via Fourier series
    \pgfmathsetmacro{\temp}{0}
    \foreach \n in {1,3,5,7,9} {
      \pgfmathsetmacro{\coeff}{4 / (3.14159 * \n)}
      \pgfmathsetmacro{\decay}{exp(-\n * \n * 3.14159 * 3.14159 * \tval)}
      \pgfmathsetmacro{\mode}{\coeff * sin(\n * 180 * \xfrac) * \decay}
      \pgfmathsetmacro{\temp}{\temp + \mode}
      \global\let\temp\temp
    }

    % Clamp temperature to [0, 1.3]
    \pgfmathsetmacro{\temp}{max(0, min(1.3, \temp))}

    % Map temperature to color: cold=atlantic, warm=garnet, hot=honeycomb
    \pgfmathtruncatemacro{\rval}{min(100, max(0, \temp * 100))}

    \fill[garnet!\rval!atlantic]
      ({\i * \dx}, 0) rectangle ({(\i + 1) * \dx}, \barheight);
  }

  % Border
  \draw[black, thick] (0, 0) rectangle (\barwidth, \barheight);

  % Temperature profile curve above the bar
  \begin{scope}[shift={(0, 1.8)}]
    \draw[->] (0,0) -- (\barwidth + 0.3, 0) node[right, font=\tiny] {$x$};
    \draw[->] (0,0) -- (0, 2.3) node[above, font=\tiny] {$T$};

    \draw[garnet, thick, smooth]
      plot[domain=0:1, samples=60, variable=\xf]
      ({\xf * \barwidth}, {%
        1.5 * (
          4/(3.14159*1) * sin(1*180*\xf) * exp(-1*3.14159*3.14159*\tval) +
          4/(3.14159*3) * sin(3*180*\xf) * exp(-9*3.14159*3.14159*\tval) +
          4/(3.14159*5) * sin(5*180*\xf) * exp(-25*3.14159*3.14159*\tval) +
          4/(3.14159*7) * sin(7*180*\xf) * exp(-49*3.14159*3.14159*\tval)
        )
      });

    % Initial condition (faint)
    \draw[warmgrey!40, dashed, smooth]
      plot[domain=0:1, samples=40, variable=\xf]
      ({\xf * \barwidth}, {%
        1.5 * (
          4/(3.14159*1) * sin(1*180*\xf) +
          4/(3.14159*3) * sin(3*180*\xf) +
          4/(3.14159*5) * sin(5*180*\xf) +
          4/(3.14159*7) * sin(7*180*\xf)
        )
      });
  \end{scope}

  % Labels
  \node[below, font=\small] at (5, -0.3)
    {$\alpha t / L^2 = \pgfmathprintnumber[fixed, precision=3]{\tval}$};
  \node[font=\tiny] at (0.8, 0.6) {Cold};
  \node[font=\tiny] at (9.2, 0.6) {Cold};

  % Colorbar legend
  \begin{scope}[shift={(11, 0)}]
    \foreach \j in {0,...,20} {
      \pgfmathtruncatemacro{\cval}{\j * 5}
      \fill[garnet!\cval!atlantic] (0, {\j * \barheight / 20})
        rectangle (0.4, {(\j + 1) * \barheight / 20});
    }
    \draw[black, thin] (0, 0) rectangle (0.4, \barheight);
    \node[right, font=\tiny] at (0.5, 0) {$0$};
    \node[right, font=\tiny] at (0.5, \barheight) {$T_{\max}$};
  \end{scope}
\end{tikzpicture}
\end{document}
